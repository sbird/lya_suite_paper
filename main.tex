% \documentclass[fleqn,usenatbib]{mnras}
\documentclass[a4paper,11pt]{article}
% \pdfoutput=1
\usepackage{amsmath, amssymb}

\usepackage{jcappub}
\usepackage{newtxtext,newtxmath}
\usepackage[T1]{fontenc}
\usepackage{graphicx}
\usepackage{hyperref}
% \usepackage{amssymb}
% Allow "Thomas van Noord" and alike to be sorted by "N" etc. in the bibliography.
% Write the name in the bibliography as "\VAN{Noord}{Van}{van} Noord, Thomas"
\DeclareRobustCommand{\VAN}[3]{#2}
\let\VANthebibliography\thebibliography
\def\thebibliography{\DeclareRobustCommand{\VAN}[3]{##3}\VANthebibliography}

% martin comment function
\newcommand{\maf}[1]{{\textcolor{red}{[{\bf MAF}: #1]}}}
% mingfeng comment function
\newcommand{\mfh}[1]{{\textcolor{green}{[{\bf MFH}: #1]}}}
% simeon comment function
\newcommand{\spb}[1]{{\textcolor{magenta}{[{\bf SPB}: #1]}}}

% ming-feng emulator commands
\newcommand{\mfemu}{\texttt{MFEmulator}}     % i made the acronym, but not used a lot
\newcommand{\Data}{\mathcal{D}}
\newcommand{\gp}{\textsc{gp}}
\newcommand{\normal}{\mathcal{N}}
\newcommand{\GP}{\mathcal{GP}}
\newcommand{\thetavec}{\boldsymbol{\theta}}  % input parameters
\newcommand{\zvec}{\boldsymbol{z}}
\newcommand{\outputFunction}{Z}
\newcommand{\outputVector}{\zvec}            % output response
\newcommand{\outputVectorModel}{\zvec_\mathrm{model}}
\newcommand{\outputVectorObs}{\zvec_\mathrm{obs}}
\newcommand{\muvec}{\boldsymbol{\mu}}        % GP mean vector
\newcommand{\Kvec}{\boldsymbol{\mathrm{K}}}  % covariance matrix
\newcommand{\kvec}{\boldsymbol{k}}           % covariance vector
\newcommand{\Lya}{Lyman-$\alpha$}
\newcommand{\astrid}{\texttt{ASTRID}}
\newcommand{\nat}{Nature}

\newcommand{\uniform}{\mathcal{U}}

\newcommand{\apjs}{ApJ Supplement}
\newcommand{\apj}{ApJ}
\newcommand{\apjl}{ApJL}

\newcommand{\aap}{AAP}
\newcommand{\mnras}{MNRAS}
\newcommand{\prd}{PRD}
\newcommand{\gadget}{{\small GADGET}}
\newcommand{\mpgadget}{{\small MP-GADGET}}

\newcommand{\km}{k_{max}}

\newcommand{\vect}[1]
  {\mbox{\boldmath ${#1}$}}
\newcommand{\matr}[1]
  {\mbox{\bf \sf{#1}}}
\newcommand{\eq}[1]
  {Eq.~(\ref{equation:#1})}
\newcommand{\eqs}[1]
  {Eqs~(\ref{equation:#1})}
\newcommand{\sect}[1]
  {section~\ref{section:#1}}
\newcommand{\sects}[1]
  {sections~\ref{section:#1}}
\newcommand{\tabl}[1]
  {{\mbox Table~\ref{table:#1}}}
\newcommand{\tabls}[1]
  {{\mbox Tables~\ref{table:#1}}}
\newcommand{\fig}[1]
  {Fig.~\ref{Figure:#1}}
\newcommand{\figs}[1]
  {Figs.~\ref{Figure:#1}}
\newcommand{\sourcesection}[1]{\noindent {\em{#1}} ---}

\def\jcap{JCAP}        % Journal of Cosmology and Astro-Particle Physics

\newcommand{\Msun}{\, h^{-1} M_\odot}
\newcommand{\Zsun}{Z_\odot}
\newcommand{\NHunit}{cm$^{-2}$}
\newcommand{\sLLS}{\sigma_\mathrm{LLS}}
\newcommand{\Mpc}{\,\mathrm{Mpc}}
\newcommand{\Mpch}{\, h^{-1} \mathrm{Mpc}}
\newcommand{\kpch}{\, h^{-1}\mathrm{kpc}}
\newcommand{\hMpc}{\, h \mathrm{Mpc}^{-1}}
\newcommand{\kms}{km~s$^{-1}$}
\newcommand{\NHI}{N_\mathrm{HI}}

\newcommand{\edit}[1]{#1}

%opening
\title{A New Suite of Lyman-$\alpha$ Forest Simulations for Cosmology}

\author[a,1]{Martin Fernandez,\note{Corresponding author}}
\author[a]{Simeon Bird,}
\affiliation[a]{Department of Physics \& Astronomy, University of California  Riverside,\\900 University Avenue, Riverside, CA 92521, USA}
\author[b]{Others: Yueying, Tiziana, Nianyi, Yu? Rupert? Anson (for the hydrogen reionization boost)?}
\emailAdd{mfern027@ucr.edu}
\emailAdd{sbird@ucr.edu}

\abstract{
We present a new suite of $43$ simulations of the Lyman-$\alpha$ forest, spanning a $9$-dimensional parameter space, including $4$ cosmological parameters and $5$ astrophysical/thermal parameters. These simulations advance on earlier simulations suites by larger particle loads, by incorporating new physical models for hydrogen and helium reionization, and by self-consistently incorporating a model for AGN feedback. Parameters are chosen based on a Latin hypercube design and a Gaussian process is used to interpolate to arbitrary parameter combinations. Our simulation suite will be used to interpret existing Lyman-$\alpha$ forest 1D flux power spectra from SDSS and future DESI data releases.
}

\begin{document}

\maketitle

\section{Introduction}

Introduce the problem. Cite the previous simulation suites for the Lyman alpha forest. The purpose of a simulation suite is to compare to observational measurements. The main output of our suite is a set of medium resolution artificial flux power spectra, from which the flux power spectrum may be generated. In this paper we describe the simulation suite we have run and defer description of the likelihood function to upcoming work.

\cite{Puchwein:2022}

\section{Simulations}

Overall, we have performed cosmological hydrodynamic simulations at two resolutions, over a parameter space containing a total of $9$ parameters describing both cosmology and astrophysics. Our lower resolution simulations, of which we have $48$, contain $1536^3$ gas particles, $1536^3$ cold dark matter particles and model a $120$ Mpc/h periodic box with a mean interparticle spacing of $79$ kpc/h. Our $3$ higher resolution simulations contain $2\times 3072^3$ particles and thus a mean interparticle spacing of $39$ kpc/h. These choices are justified in Section~\ref{sec:boxsize}. All simulations were run to $z=2.2$, the lowest redshift at which SDSS measures the \Lya~forest flux power spectrum.

Simulations were run using \mpgadget~\cite{MPGadget2018, Bird:2022}, a massively scalable version of the cosmological structure formation code Gadget-3 \cite{Springel:2005}. The generation of initial conditions is described in Section~\ref{sec:initconds}. Section~\ref{sec:gravity} describes the gravity solver, hydrodynamic solver, and cooling.

Sections~\ref{sec:hydrogen} and ~\ref{sec:helium} describe our models for hydrogen and helium reionization respectively, and how their parameters map to the familiar thermal parameters of the \Lya~forest. We have chosen to vary the parameters of our hydrogen and helium reionization models in order to model uncertainty in the thermal history of the intergalactic medium (IGM).
A wide variety of earlier models instead generate arbitrary thermal histories by scaling the heating rates directly \cite[e.g.~][]{Viel:2006}. Scaling the heating rates allows the greatest possible freedom in the redshift evolution of the thermal histories, but does does not directly impose a physically plausible thermal history, which can sometimes lead to best-fit thermal histories which do not correspond to actual simulations \cite{Walther:2019}, Conversely, Ref.~\cite{Garzilli:2019} argued that the assumption of a power law evolution of the mean temperature with redshift \cite{Viel:2013wdm, Irsic:2017} provides a strong prior, although see Ref.~\cite{Garzilli:2021} where stronger data validated the original thermal evolution. Constraining the parameters of physical reionization models ensures that the redshift evolution of the thermal histories produced are physically plausible and self-consistent, and thus simplifies our inference. This choice is enabled by two advances: first, our simulation boxes are large enough to contain many helium reionization bubbles and second, the thermal history of the IGM is now reasonably well constrained, limiting the variety of models it is necessary to marginalise over.

We use a suite of full physics simulations including star formation, stellar winds and AGN feedback. We use full physics simulations primarily because it allows us to incorporate our AGN feedback model self-consistently into the code, rather than with a post-processing correction. AGN feedback has been shown to influence the \Lya~forest flux power spectrum \cite{Viel:2013, Chabanier:2020}. Full-physics simulations also allow us to self-consistently include the effect of self-shielded gas in the modelling, by producing DLAs in our spectral code and removing them, as in the observational analysis. We stress that these are likely small effects and we do not wish to imply that quick \Lya~models are inadequate for current data. However, with current simulation codes the speed difference between the quick \Lya~and full physics simulations is only $\sim 30\%$, and so the cost is relatively small. Furthermore, using full physics simulations allows our suite to be used for other applications in future.

Our galaxy formation model parameters generally follow those used in the \astrid~simulation, as detailed in Refs.~\cite{Bird:2022, Ni:2021}. Here we summarise the main features of the model, and refer the interested reader to \cite{Bird:2022} for more details. Readers familiar with \astrid~should refer to Table~\ref{tab:paramchanges}, which summarises the differences in the galaxy formation model. We describe the star formation and stellar wind model in Section~\ref{sec:stellar}, and the black hole and AGN feedback model in Section~\ref{sec:agn}.

\begin{table*}
\begin{centering}
  \begin{tabular}{lll}
  \hline
  Parameter & New Value & \astrid~Value \\
    \hline
SPH density kernel  & Cubic & Quintic \\
Minimum wind velocity $v_w$ & $100$ km/s & $0$ \\
Temperature at HI reionization & $15000$ K & $0$ K \\
Stars per gas particle & $1$ & $4$ \\
Metal return from stars & Disabled & Enabled\\
Metal line cooling & Disabled & Enabled. \\
    \hline
  \end{tabular}
  \caption{Summary of changes and simplifications in the galaxy formation model from the \astrid~simulation.}
  \label{tab:paramchanges}
  \end{centering}
\end{table*}

%The initial conditions include separate transfer functions for baryons and cold dark matter. The initial distribution of cold dark matter particles is a grid, and the initial distribution of baryons is a Lagrangian glass, as recommended by \cite{Bird:2020} to reduce particle transients. First-order Lagrangian perturbation theory \citep{Zeldovich:1970, Crocce:2006} is used to initialise particle positions and velocities. Second-order terms are not included as they are difficult to model when the initial transfer functions for baryons and cold dark matter are different. Radiation density is included in the cosmological background evolution.
%, a $10\%$ contribution to the Hubble expansion at $z=100$ that affects the total integrated growth rate at the $10\%$ level by $z=10$.

\subsection{Initial Conditions}
\label{sec:initconds}

We generate initial condition for our simulations as in Ref.~\cite{Bird:2020}. We use separate transfer functions for gas and cold dark matter particles, as generated by CLASS \cite{CLASS}. Velocities are initialised using the respective velocity transfer functions, so that the variation in halo gas fraction from dark matter-baryon relative velocities are automatically included in the simulation. Radiation is included in the cosmological background evolution. Simulations are initialised at $z=99$. Cold dark matter is initialised on a regular grid and gas particles are initialised using a force-free Lagrangian glass.

Our emulator is designed to constrain cosmological parameters and so we perform simulations varying cosmological parameters in the initial conditions. Specifically, we define the primordial matter power spectrum as:
\begin{equation}
 P(k) = A_P \left(\frac{k}{ 0.78 \mathrm{h/Mpc}}\right)^{n_s-1}\,.
 \label{eq:pk}
\end{equation}
Here $n_s$ is the spectral index. $A_p$ is the amplitude of perturbations at $k = 0.78$ h/Mpc, chosen to be in the middle of the range probed by the \Lya~forest and thus reduce artificial correlation between $n_s$ and $A_P$ \cite{Bird:2019}. We vary $A_P$ in the range $1.2 \times 10^{-9}$ and $2.6 \times 10^{-9}$, chosen the range of $A_P$ simulated to include the posterior constraint from Planck on larger scales \cite{Planck:2018}.
$n_s$ is the spectral index measured on small scales, which we simulate between $0.8$ and $0.995$. Note that in extended models with a scale-dependence of the spectral index (a ``running''), this need not have the same value as the spectral index measured by Ref.~\cite{Planck:2018}.

We do not include a parameter for massive neutrinos. Neutrinos are massless and included in the radiation, as the effect of massive neutrinos on these scales is completely degenerate with the amplitude of the initial perturbations, $A_P$ \cite{Pedersen:2020}.
Schematically, neutrino mass constraints come from measuring different power spectrum amplitudes on CMB and \Lya~scales.

We also include free parameters for the growth rate and Hubble parameter, although these are generally measured better by other probes than the \Lya~forest. However, our Latin Hypercube experimental design (see Section~\ref{sec:latinhypercube}) allows us to vary extra parameters at low computational cost, if they only weakly affect the flux power spectrum.
We vary $h$ over the range $0.65$ - $0.75$ and $\Omega_0 h^2$ between $0.14$ and $0.146$. In our final cosmological analysis we will impose priors from Planck on $\Omega_0 h^2$.
We do not vary $\Omega_b$ as it is degenerate with the ionization fraction of the gas and thus the mean optical depth $\tau$.

\subsection{Gravity, Hydrodynamics and Cooling}
\label{sec:gravity}

%The gravity solver in MP-Gadget uses the same basic algorithms as Gadget-3.
We use the same gravity solver described in Ref.~\cite{Bird:2022} and references therein. Long range gravitational forces are computed in Fourier space using a particle-mesh algorithm. Short-range forces, below the resolution of the particle grid, are computed using a hierarchical multipole expansion of the gravitational field, leading to a uniformly high force resolution throughout the computational volume. The default accuracy of the split between short and long range forces has been increased and the force accuracy of \mpgadget~is better than 1\% as measured against an exact N-body solver.
%able to reproduce the matter power spectrum output by PKDGRAV3 to $<1\%$ \cite{Schneider:2016}.

We adopt the pressure-entropy formulation of smoothed particle hydrodynamics (pSPH) to solve the Euler equations \citep{Hopkins:2013,Read:2010} as implemented by \cite{Feng:2014}. %The density estimator uses a quintic density kernel to reduce noise in SPH density and gradient estimation \citep{2012JCoPh.231..759P}.
The SPH density kernel uses a cubic polynomial, rather than the quintic kernel from \astrid, as the reduced number of neighbours improves resolution in the \Lya~forest. This comes at the cost of increased noise in dense regions, but at forest densities this does not affect the dynamics \cite{Bird:2013}.
Gas is allowed to cool radiatively following \citep{Katz:1996}.
We have used the updated cooling coefficients summarised in \cite{Bolton:2017}, the recombination rates from \cite{Verner:1996} and collisional ionization rates from \cite{Voronov:1997}. Self-shielding of neutral hydrogen is included following the fitting function of \cite{Rahmati:2013}.
The UV background used is the ``optically thin'' variant from Ref.~\cite{FG2020}. There is also a variant which corrects for the average effect of partial reionization on the gas temperature; we do not use this as we instead include explicit models for patchy reionization.

\subsection{Hydrogen Reionization}
\label{sec:hydrogen}

We model patchy reionization with a spatially varying ultra-violet background using a semi-analytic method \cite{Battaglia:2013} which pre-computes a reionization redshift, on a grid with cells $1$ Mpc/h, using FASTPM \cite{FASTPM}. The redshift of reionization, $z_{HI}$, is defined as the redshift at which $50\%$ of the simulation volume has reionized. In general, higher initial overdensities reionize earlier. We simulate $z_{HI}$ over the range $6.5 - 8$. Our use of a patchy reionization model means that we naturally include lower redshift effects of reionization on the \Lya~forest \citep{Montero:2019}

For gas particles in a region which has not yet reionized, the photon background is set to zero and the gas remains neutral. Once the reionization redshift has passed, the particle's ionization and thermal states evolve in ionization equilibrium with the external UV background. Newly in these simulations, we include a temperature boost to each gas particle as it passes the reionization redshift to account for the (subgrid) heating effect of ionization fronts.
Each gas particle on reionization has its temperature boosted to $15,000$K.
Ref.~\citep{DAloisio:2019} found a temperature boost of $\sim 20,000$K in their radiative transfer simulations. However, our simulation is much lower resolution and has much longer timesteps, suggesting that we should expect a certain amount of subgrid cooling before any given particle in our simulation box becomes active. Furthermore, we found that a temperature boost of $20,000$ K led to generally higher IGM temperatures when compared to observations $z\sim 6$, at least for the relatively low reionization redshifts preferred by current mean free path observations \cite{Cain:2021}.

The reionization temperature boost means that $z_{HI}$ affects the thermal history of the IGM. After reionization the UV background is not sufficiently intense to preserve a thermal equilibrium and so the gas cools. The reionization parameter, $z_{HI}$, thus controls the IGM temperature before helium reionization starts at at $z \gtrsim 4$. Note that as long as $z_{HI} > 5.8$, the current highest redshift IGM temperature measurement \cite{Gaikwad:2020}, $z_{HI}$ is observationally degenerate with the size of the reionization temperature boost. Our ultimate constraints on the reionization redshift should thus be understood to be conditional on a boost of $15,000$ K.

%We do not boost the gas temperature to account for the passage of ionization fronts \citep{2019ApJ...874..154D}.

\subsection{Helium Reionization}
\label{sec:helium}

We include a model for spatially inhomogeneous helium reionization following \cite{UptonSanderbeck:2020}. Reionized bubbles are placed around randomly chosen halos with mass $10^{13} \geq M_{\rm halo}\geq 10^{12}$M$_{\odot}$, which could host a quasar. Bubbles are $20$ Mpc/h in radius, matching the mean bubble size found in radiative transfer simulations \citep{McQuinn:2009}. Inside this bubble a quasar radiation field is assumed with an effective spectral index of $- \alpha_q$, which we vary in our simulation suite over the range $\alpha_q  = 1.6 - 2.5$. During the reionization timestep gas particles in a bubble are abruptly heated and marked as ionized. Should bubbles overlap, only particles not already marked as ionized are heated. Bubbles are placed until the total ionized gas mass in the box reaches
a pre-computed ionization fraction. We assume a linear ionization history between the initial redshift $z_{Hei}$ and the final redshift $z_{Hef}$. Following measurements of the optical depth in the Helium \Lya~forest \cite{Worseck:2019}, we simulate $z_{Hei} = 3.5 -  4.1$ and $z_{Hef} = 2.6 - 3.2$.

Our $120$ Mpc/h box simulations contain $\sim 50$ reionization bubbles and so our model includes the patchiness of helium reionization and the resulting fluctuations in temperature, ionization state, and pressure smoothing. A consequence is that there is no unique temperature-density relation in our simulations, but a range of temperature-density relations depending on the time at which a given part of the simulation reionized. Our three free parameters allow us to generate a wide range of thermal histories, as shown in Figure~\ref{fig:heliumtempdens}. The quasar spectral index, $\alpha_q$ controls the level of heating during helium reionization and thus the peak IGM temperature. Higher values of $\alpha_q$ lead to lower peak temperatures.

%$z_{Hei}$ and $z_{Hef}$ are the start and end redshifts, respectively, of helium reionization. $z_{Hi}$ is the midpoint redshift of hydrogen reionization. $\alpha_q$ is the quasar emissivity spectral index, which controls the peak IGM temperature at the midpoint of helium reionization.

\begin{table*}
\begin{centering}
  \begin{tabular}{llll}
  \hline
  Parameter & Minimum & Maximum & Description \\
    \hline
    $n_s$  &  $0.8$  & $0.995$ & Scalar spectral index \\
    $A_P$  &  $2.2 \times 10^{-9}$  & $2.6 \times 10^{-9}$ & Power amplitude at $k = 0.78$ h/Mpc \\
    $h$    & $0.65$  & $0.75$ & Hubble parameter \\
    $\Omega_0 h^2$ & $0.14$ & $0.146$ & Total matter density \\
    $z_{Hei}$      & $3.5$  & $4.1$  & Start redshift of HeII reionization \\
    $z_{Hef}$      & $2.6$  & $3.2$  & End redshift of HeII reionization \\
    $\alpha_q$     & $1.6$  & $2.5$ & Quasar spectral index during HeII reionization  \\
    $z_{Hi}$        & $6.5$ & $8$   & Median redshift of HI reionization \\
    $\epsilon_{AGN}$ & $0.03$ & $0.07$ & Thermal efficiency of black hole feedback \\
    \hline
  \end{tabular}
  \caption{Summary of varied emulator parameters, together with the ranges covered by the emulator. We vary a total of $9$ parameters: $4$ for cosmology, $3$ for the helium reionization model, $1$ for the hydrogen reionization model and $1$ for the strength of AGN feedback.}
  \label{tab:emulatorparams}
  \end{centering}
\end{table*}

\subsection{Star Formation and Stellar Feedback}
\label{sec:stellar}

Stars form on an effective equation of state in dense gas, following \cite{Springel:2003}. Stars are formed with the same mass as a gas particle, so that gas particles are converted directly to stars. Note that \astrid~formed stars with $1/4$ of the mass of the gas particle. This increases the number of particles in dense regions which have little effect on the \Lya~forest, and so for our simulation suite simply increases computational cost.

We disable metal return from stars, as evaluating this model can be computationally expensive, and we are building an emulator for the neutral hydrogen of the \Lya~forest. We also disable metal cooling, which Ref.~\cite{Viel:2013} showed has a small effect on the \Lya~forest flux power spectrum. Similarly, we do not include the (metallicity dependent) correction to the high redshift star formation rate due to the formation of molecular hydrogen implemented in \astrid, which has a very small effect for $z < 6$.

A stellar wind feedback model is included following Ref.~\citep{Okamoto:2010} and Ref.~\cite{Bird:2022}. Wind speeds are proportional to the local one dimensional dark matter velocity dispersion $\sigma_\mathrm{DM}$:
\begin{equation}
v_w = \kappa_w \sigma_\mathrm{DM} \,,
\end{equation}
where $v_w$ is the wind speed. $\kappa_w$ is a dimensionless parameter, which we take to be $3.7$ following \cite{Vogelsberger:2013}. We have mildly altered the model from \cite{Bird:2022} by implementing a minimum wind velocity, $v_w \geq 100$ km/s, in order to improve convergence with resolution.

Winds are sourced by newly formed star particles, which randomly pick gas particles from within their SPH smoothing length to become wind particles. The total mass loading is $(v_w/ 350 \mathrm{km/s})^{-2}$ where $350$ km/s is in physical units. Particles recouple when their surrounding density drops by a factor of $10$, or after $60$ Myr.\footnote{In Ref.~\cite{Bird:2022} we had a subdominant recoupling condition: gas would recouple after $20 \mathrm{kpc} / v_w$. In practice this affected only a small fraction of the wind, as for a typical star forming halo with a virial velocity of $200$ km/s the recoupling time was $100$ Myr, and gas always recoupled after $60$ Myr. We have thus removed this condition for model simplicity.} Particles in the wind cool, but do not experience or produce pressure forces, nor may they be accreted onto a black hole. However, to enhance the stability of the SPH density estimates, they are included when computing SPH smoothing lengths.

We conducted an early test emulator using small boxes where the dimensionless supernova wind velocity, $\kappa_w$, was a free parameter of the model. However, we found that this free parameter was essentially unconstrained by the \Lya~forest data and so opted not to vary it in the full emulator run. Ref.~\cite{Bolton:2017} showed that supernova winds increased the \Lya~flux power on large scales, $k < 10^{-2}$ s/km, at $z < 3$, by around $10\%$, due to the presence of additional high column density systems\footnote{Ref.~\cite{Viel:2013} found that the \Lya~flux power spectrum was increased only for $ k > 0.04$ s/km (scales smaller than those measured by BOSS), but their simulations did not include self-shielding and so did not contain high column density absorbers.}. We remove high column density systems from our simulated spectra to match the observational procedure, and thus changes to the supernova wind model do not affect the \Lya~flux power spectrum. As shown in Figure~\ref{fig:DLACDDF}, our supernova simulation model parameters have been chosen to match the observed galaxy stellar mass function and thus the distribution of high column density absorbers. This good agreement can be taken as a strong prior on the supernova wind parameters and justifies our choice to fix the wind model in our emulator.

\begin{figure*}
\centering
\includegraphics[width=0.6\textwidth]{figures/cddf_hires.pdf}
 \caption{Column density distribution function (CDDF) of high column density absorbers from our $3$ high resolution simulations at $z=2.2$, compared to the CDDF from SDSS DR16 of Ref.~\cite{2021MNRAS.507..704H}}.
 \label{fig:DLACDDF}
\end{figure*}

\subsection{Black Holes and AGN Feedback}
\label{sec:agn}

Our simulations include super-massive black hole (SMBH) seeding, growth and feedback following Ref.~\cite{Ni:2022}, itself based on \cite{Feng:2016,SDH2005,DSH2005}. Compared to \astrid, we have adapted some of the thresholds to our lower resolution simulations.

SMBH particles are seeded by converting the densest gas particle found in a halo. To be eligible for seeding, a halo must have total mass greater than $5\times 10^{10}\,h^{-1}M_\odot$ and stellar mass greater than $2 \times 10^9 h^{-1} M_\odot$. The SMBH seed mass is $5 \times 10^{-5} h^{-1} M_\odot$. The initial dynamic mass of the SMBH, which keeps the dynamical friction model stable for BH seeds, is set as $10^{8} h^{-1} M_\odot$. For comparison, the gas particle mass is $5 \times 10^6 h^{-1} M_\odot$ and the CDM mass is $ 3 \times 10^7 h^{-1} M_\odot$. SMBH are thus seeded only in well-resolved halos with (on average) $> 1400$ CDM particles and $>400$ star particles.

We implement a model for dynamical friction following Ref.~\citep{Chen:2021}. This ensures that SMBHs are kept close to the centers of their halos, and replaces the earlier manual repositioning. Dynamical friction is an artificial force modelling unresolved small-scale interactions between the SMBH and nearby stars. These interactions transfer momentum from the SMBH to individual stars in the surrounding star clusters. We include dynamical friction from star and CDM particles, but in practice the star particles strongly dominate \cite{Chen:2021}.

BHs are allowed to grow by accreting mass from nearby gas particles. Gas accretes following the Bondi-Hoyle-Lyttleton formula, applied to the smoothed properties of the gas particles within the SPH kernel of the BH:
\begin{equation}
\label{equation:Bondi}
    \dot{M}_{\rm B} = \frac{4 \pi \alpha G^2 M_{\rm BH}^2 \rho}{(c^2_s+v_{\rm rel}^2)^{3/2}}
\end{equation}
$c_s$ and $\rho$ are the local sound speed and density of gas, $v_{\rm rel}$ is the relative velocity of the BH with respect to the nearby gas and $\alpha = 100$ is a dimensionless fudge parameter to account for the underestimation of the accretion rate due to the unresolved cold and hot phase of the subgrid interstellar medium nearby.
Note that hydrodynamically decoupled wind particles are not included in the density calculation of Eq.~\ref{equation:Bondi}. The accretion rate is capped at $2$ times the Eddington accretion rate.
%Numerically, the physical BH mass, $\mbh$, grows continuously with time, while we separately track the BH dynamic mass by stochastic accretion of nearby gas particles with a probability that statistically satisfies mass conservation. Given that $M_{\rm dyn}$ is temporarily boosted (and initially fixed) for the new-born BH particles for gravity and dynamical friction calculation, we use a separate mass tracer $M_{\rm trace}$ to perform gas accretion when $\mbh < M_{\rm dyn,seed}$.
%This term is initialized as the parent particle mass that spawns the BH, and traces $\mbh$ with stochastic gas accretion until $\mbh >= M_{\rm dyn,seed}$. After that, $M_{\rm dyn}$ grows following $\mbh$ by swallowing gas.
%To ensure stable accretion, the timestep of the BH particle is set not by the acceleration dynamics, but by being $2$ times longer than the smallest timestep found amongst the gas particles within its density kernel.

AGN thermal feedback energy is implemented by depositing a fraction, $\epsilon_{AGN}$ of the available radiative energy for the AGN into the gas. We assume an accretion disk with a mass-to-light conversion efficiency of $0.1$, so that the AGN thermal energy is \cite{Shakura:1973}
\begin{equation}
\label{equation:Lbol}
    E_\mathrm{AGN} = \frac{\epsilon_{AGN}}{10} \dot{M}_{\rm BH} c^2\,.
\end{equation}
The efficiency of the thermal feedback, $\epsilon_{AGN}$ is a free parameter of our emulator and simulate it in the range $0.03 - 0.07$.

$E_\mathrm{AGN}$ is deposited isotropically to particles within $3$ kpc/h. We use $3$ kpc/h instead of the SPH smoothing radius in an attempt to improve convergence with resolution, as this is roughly the smoothing length at the resolution of \astrid. However, in practice this choice has no measurable effect on the \Lya~forest. The AGN feedback modifies the cooling time of the star-forming gas, allowing it to cool on the cooling timescale rather than the longer relaxation timescale, and thus moderately increases the star formation rate.

In our purely thermal AGN feedback model, AGN feedback does not remove gas from a halo and thus does not suppress the matter power spectrum. At $z > 2$ the matter power spectrum suppression is fairly mild in AGN feedback models, as there are few SMBHs in the low accretion regime necessary to induce kinetic mode feedback. While the effect of AGN feedback on weak lensing is driven by the removal of gas from massive halos \cite[e.g.~][]{Giri:2021}, the \Lya~forest is also directly sensitive to the gas temperature around halos, and so may be affected by purely thermal feedback.

Our emulator design also allows us to measure parameter degeneracies between AGN feedback and cosmological parameters. Ref~\cite{Schneider:2015} showed a degeneracy between AGN feedback and the fraction of matter in baryons, $\Omega_b / \Omega_0$. However, since the \Lya~forest may be sensitive to the gas temperature, there may also be a degeneracy with the number density of actively radiating SMBHs, which we would be able to detect.

%\cite{Giri:2021} show that the effect is well described by the halo scale at which AGN feedback starts to remove the gas.
%Since the level of AGN feedback present in simulations is currently uncertain, we vary the strength of thermal feedback in our emulator.

\subsection{Generation of Spectra}
\label{sec:spectra}

We generate artificial spectra and compute the flux power spectrum as described in \cite{Bird:2019}. Each simulation generates output snapshots evenly spaced every $\Delta z = 0.2$ between $z = 5.2$ and $z = 2$. Spectra are generated using ``fake\_spectra'' \footnote{\url{https://github.com/sbird/fake_spectra}} \cite{FSFE:2017}. We generate $32000$
\Lya~absorption spectra at random positions. To improve emulation performance, the random positions are the same for each simulation. The pixel width is $10$ km/s, finer than either the BOSS or DESI spectrograph. Flux power spectra are generated in h/Mpc units, with the natural Fourier bins of the box, and transformed to velocity units after emulation. The flux power spectrum is generated as average of the 1D Fourier transform along each sightline.

We rescale our simulated spectra to match a desired mean optical depth. As in \cite{Bird:2019}, our emulator has the ability to generate flux power spectra for arbitrary input mean optical depth $\tau$. This is achieved by rescaling the spectra in post-processing: no extra simulations are needed. We are thus able to dramatically over-sample the mean optical depth as an input parameter, generating a dense grid of $10$ optical depth samples per redshift per cosmological simulation. As explained in \cite{Bird:2019}, we generate one Gaussian Process emulator for every redshift bin, each emulator taking the mean optical depth at that redshift as an input parameter. We would thus be able to use very general redshift variations of the mean optical depth if supported by observational data.

% This is done because for any given redshift bin there is a degeneracy between $\tau_0$ and $d\tau_0$. Computing the mean flux in each redshift bin separately breaks this degeneracy and avoids duplicate spectra.


\subsection{Experimental Design}
\label{sec:latinhypercube}

\begin{figure}
    \centering
	\includegraphics[width=0.5\columnwidth]{figures/120box_42samples.pdf}
    \caption{Simulation parameter limits, and parameters simulated. Shown as crosses are the $30$ initial simulations chosen from a Latin hypercube, while the squares show the secondary set of $13$ simulations chosen using Bayesian optimisation.
    }
    \label{fig:samples}
\end{figure}

% Describe the chosen cosmological parameters. Add a plot as Figure 3 of the multi-fidelity paper. Explain why these parameters are chosen. We use a Latin Hypercube design. Outputs are every $\Delta z=0.2$ from $z=5.4$ to $z=2.0$.

Our emulator includes a total of $10$ parameters at each redshift. One parameter, the mean flux, is generated in post-processing, meaning that we run simulations covering a $9$-dimensional parameter space, summarised in Table~\ref{tab:emulatorparams}.

We generate simulations at specific points in parameter space using a Latin Hypercube design, following the description in \cite{Bird:2019}. Latin Hypercubes are generated at random on a normalised unit cube, and the design which maximises the spread between parameter points is chosen for the final emulator. We ran $30$ simulations in our initial Latin Hypercube design. We supplemented these simulations with $2$ chosen by Bayesian optimisation, as described in \cite{Rogers:2019}. However, we found that the initial space was not finely sampled enough for Bayesian Optimization to be effective: we thus ran an extra $8$ simulation Latin Hypercube, before $3$ more simulations chosen with Bayesian Optimisation. \spb{What was the acquisition function used?} The parameters simulated are shown in Figure~\ref{fig:samples}, which demonstrates that both sets efficiently cover parameter space.

\subsection{Multi-Fidelity Emulator}

Finally, we construct a multi-fidelity Gaussian Process emulator following the scheme described in \cite{Bird:2019, Fernandez:2022}. The covariance kernel is the combination of a radial basis function and a dot kernel. The length-scale of the radial basis function for each parameter is a hyper-parameter chosen by optimisation.
We use a multi-fidelity emulation scheme following \cite{Fernandez:2022}. We use the AR1 emulation model, where the resolution correction is modelled using a correction function, $\rho$, which is a function of $k$, $z$ and cosmological parameter. If the target high fidelity flux power spectrum is $P_{F, HF}$ and the low fidelity flux power spectrum is  $P_{F, LF}$, we thus correct for resolution using
\begin{equation}
 P_{F, HF}(k, z, x) =  \rho(x) P_{F, LF}(k,z,x) + \delta(k, z, x)
\end{equation}
\spb{Expand and correct this}

We have thus built a surrogate model able to target a mean interparticle spacing of $39$ kpc/h.

\section{Results}

\subsection{Box Size and Particle Load}
\label{sec:boxsize}

\begin{figure*}
\includegraphics[width=0.95\textwidth,trim={0 7cm 1cm 0},clip]{figures/box-convergence.pdf}
 \caption{Convergence of the flux power spectrum with box size. Shown is the ratio of the flux power spectrum as a function of redshift in a $120$ Mpc/h box compared to a $60$ Mpc/h box. Our primary simulations use a $120$ Mpc/h box. This is thus an upper limit on the convergence with respect to the simulation box. The range of $k$ measured by eBOSS is $10^{-3}$ s/km to $2\times 10^{-2}$ s/km. Other redshift bins are similar.}
 \label{fig:boxsize}
\end{figure*}

Our lower resolution simulations, of which we have $48$, contain $1536^3$ gas particles, $1536^3$ cold dark matter particles and model a $120$ Mpc/h periodic box with a mean inter-particle spacing of $79$ kpc/h. Our $3$ higher resolution simulations contain $2\times 3072^3$ particles and thus a mean inter-particle spacing of $39$ kpc/h.

Figure~\ref{fig:boxsize} shows the effect of the finite box size on the flux power spectrum, as a function of redshift. We show the flux power spectrum ratio between a test simulation in a $60$ Mpc/h simulation box and a larger $120$ Mpc/h simulation box. The mean inter-particle spacing is $120$ kpc/h for both simulations, so that the large box simulation has $1024^3$ particles, while the small box simulation has $512^3$ particles. Since our main simulations use $120$ Mpc/h simulation boxes, this is an upper limit on the effect of the finite box. The non-linear scale at $z=2$ is \spb{XX}. However, the \Lya~forest is predominantly sensitive to gas from $1-10$ times the mean density, and so the effect of non-linearities is reduced.

Cosmic variance is suppressed when compared to emulators of the matter power spectrum as our summary statistic is the $1$-D flux power spectrum, evaluated along the line of sight to the quasar and averaged over multiple quasar positions. Effectively this sums over two dimensions of the box, ensuring that even the largest scales sample Fourier modes proportional to the number of spectra. Since the larger box contains $8\times$ volume of the small box, Figure~\ref{fig:boxsize} is also a demonstration of the small effect of cosmic variance.



Describe the resolution and box size of the simulation. Justify this by citing the relevant 2014 paper and with the figure. \spb{Note that the convergence at high redshift is a little better than Bolton and Becker 2010, because the heating effect of the ionization fronts smooths the high redshift gas and prevents the small unresolved halos from accreting anyway.}

\begin{figure*}
\includegraphics[width=0.95\textwidth,trim={0 0 1cm 0},clip]{figures/resolution-convergence.pdf}
 \caption{Convergence of the flux power spectrum with resolution (mean interparticle spacing). The simulations are run with the same box size but different particle loads. The lines labelled $78$ kpc/h (blue solid) compares two simulations in a $120$ Mpc/h box. The first has $1536^3$ particles, and thus a mean-interparticle spacing of $78$ kpc/h, while the second has $3072^3$ particles. The lines labelled $39$ kpc/h compare two simulations in a $15$ Mpc/h box. The first has $384^3$ particles, and the second has $512^3$ particles.}
 \label{fig:resolution}
\end{figure*}

\subsection{Realised Thermal Histories}
\spb{Mean temperature plot for the helium reionization model parameters.}

\begin{figure*}
\includegraphics[width=0.45\textwidth]{figures/mean-temperature.pdf}
\includegraphics[width=0.45\textwidth]{figures/mean-temperature-resolution.pdf}
 \caption{(Left) Temperatures at mean density as a function of redshift for simulations in our emulator. Observational points shown are the combined $T_0$ measurements from \protect\cite{Gaikwad:2021}.  The grey band shows the range of temperatures realised by simulations in our emulator. The black line shows an example thermal history, for a simulation with reionization parameters $\alpha_q = 1.795$, $z_{Hei} = 3.79$, $z_{Hef} = 3.09$, $z_{Hi} = 7.1$. Other parameters are $n_s = 0.82275$, $A_P = 1.92 \times 10^{-9}$, $\Omega_0 h^2 = 0.1445$, $h = 0.67$, $\epsilon_{AGN} = 0.067$.
 %[8.22750000e-01 1.92333333e-09 3.79000000e+00 3.09000000e+00
 %1.79500000e+00 6.85000000e-01 1.44500000e-01 7.07500000e+00
 %6.66666667e-02]
 (Right) Ratio between the IGM mean temperature in low resolution simulations to the high resolution simulations. Convergence is at the level of $1\%$ while the observational error on the temperature is $5-10\%$.
}
 \label{fig:heliumtempdens}
\end{figure*}

Ref.~\cite{Chabanier:2022} showed that SPH and grid codes agree to less than $1\%$.

\subsection{Effect of Parameters on the Flux Power Spectrum and Temperature}

Plots of the effect of each parameter on the flux power spectrum from the emulator.

\begin{figure}
    \centering
	\includegraphics[width=0.48\columnwidth]{figures/single_param_Ap.pdf}
		\includegraphics[width=0.48\columnwidth]{figures/single_param_ns.pdf}
% 	\includegraphics[width=0.3\columnwidth]{figures/single_param_tau0.pdf}

    \caption{Effect of changing each simulation parameter on the 1D flux power spectrum at $z=2$ and $z=3$. Plots should look a bit like Figure 9 of \url{https://arxiv.org/pdf/1401.6472.pdf}. Each panel should have 5 lines. Let's have $9$ panels, 1 for each parameter. cosmology parameters ($n_s$, $A_s$, $\Omega_m h^2$, $h$). reionization parameters ($He_i$ and $He_f$, $\alpha_q$, $HI_z$). Finally black hole feedback. }
    \label{fig:fluxpower}
\end{figure}

\begin{figure}
    \centering
	\includegraphics[width=0.48\columnwidth]{figures/single_param_tau0.pdf}
    \includegraphics[width=0.48\columnwidth]{figures/single_param_dtau0.pdf}
% 	\includegraphics[width=0.3\columnwidth]{figures/single_param_tau0.pdf}

    \caption{Effect of changing each simulation parameter on the 1D flux power spectrum at $z=2$ and $z=3$. Plots should look a bit like Figure 9 of \url{https://arxiv.org/pdf/1401.6472.pdf}. Each panel should have 5 lines. Let's have $9$ panels, 1 for each parameter. cosmology parameters ($n_s$, $A_s$, $\Omega_m h^2$, $h$). reionization parameters ($He_i$ and $He_f$, $\alpha_q$, $HI_z$). Finally black hole feedback. }
    \label{fig:fluxpower}
\end{figure}

\begin{figure}
    \centering
	\includegraphics[width=0.48\columnwidth]{figures/single_param_herei.pdf}
    \includegraphics[width=0.48\columnwidth]{figures/single_param_heref.pdf}
% 	\includegraphics[width=0.3\columnwidth]{figures/single_param_tau0.pdf}

    \caption{Effect of changing each simulation parameter on the 1D flux power spectrum at $z=2$ and $z=3$. Plots should look a bit like Figure 9 of \url{https://arxiv.org/pdf/1401.6472.pdf}. Each panel should have 5 lines. Let's have $9$ panels, 1 for each parameter. cosmology parameters ($n_s$, $A_s$, $\Omega_m h^2$, $h$). reionization parameters ($He_i$ and $He_f$, $\alpha_q$, $HI_z$). Finally black hole feedback. }
    \label{fig:fluxpower}
\end{figure}

\begin{figure}
    \centering
	\includegraphics[width=0.48\columnwidth]{figures/single_param_hireionz.pdf}
    \includegraphics[width=0.48\columnwidth]{figures/single_param_alphaq.pdf}
% 	\includegraphics[width=0.3\columnwidth]{figures/single_param_tau0.pdf}

    \caption{Effect of changing each simulation parameter on the 1D flux power spectrum at $z=2$ and $z=3$. Plots should look a bit like Figure 9 of \url{https://arxiv.org/pdf/1401.6472.pdf}. Each panel should have 5 lines. Let's have $9$ panels, 1 for each parameter. cosmology parameters ($n_s$, $A_s$, $\Omega_m h^2$, $h$). reionization parameters ($He_i$ and $He_f$, $\alpha_q$, $HI_z$). Finally black hole feedback. }
    \label{fig:fluxpower}
\end{figure}

\begin{figure}
    \centering
	\includegraphics[width=0.48\columnwidth]{figures/single_param_hub.pdf}
    \includegraphics[width=0.48\columnwidth]{figures/single_param_omegamh2.pdf}
    \caption{Effect of changing each simulation parameter on the 1D flux power spectrum at $z=2$ and $z=3$. Plots should look a bit like Figure 9 of \url{https://arxiv.org/pdf/1401.6472.pdf}. Each panel should have 5 lines. Let's have $9$ panels, 1 for each parameter. cosmology parameters ($n_s$, $A_s$, $\Omega_m h^2$, $h$). reionization parameters ($He_i$ and $He_f$, $\alpha_q$, $HI_z$). Finally black hole feedback. }
    \label{fig:fluxpower}
\end{figure}

\begin{figure}
    \centering
	\includegraphics[width=0.48\columnwidth]{figures/single_param_t0.pdf}
    \includegraphics[width=0.48\columnwidth]{figures/single_param_bhfeedback.pdf}

    \caption{Effect of changing each simulation parameter on the IGM mean temperature as a function of redshift.}
    \label{fig:meantemp}
\end{figure}


\subsubsection{Effect of AGN Feedback}

Ref.~\cite{Viel:2013} showed an effect of AGN feedback which rises to $10\%$ at $z=2$, while Ref.~\cite{Chabanier:2020} found $8\%$ at $z=2$ and $k = 0.005$ s/km. Both simulations use AGN feedback models with parameters tuned to match the properties of galaxies at $z=0$. However, the implementations are different. Ref.~\cite{Chabanier:2020} uses the Horizon-AGN simulation \cite{Dubois:2016}. The AGN feedback model has two modes, one thermal and one kinetic, with the kinetic mode dominating at low accretion rates. Ref.~\cite{Viel:2013} uses the OWLS simulation suite, and delivers AGN feedback energy thermally, once black hole accretion has accumulated enough energy to heat a nearby gas particle to $10^8$ K. Despite the different implementations, the two codes agree reasonably well as to the effect of AGN feedback on the \Lya~forest.

\spb{Plot of the effect of the AGN feedback parameter. Ideally I would compare this to the correction factor from Chabanier et al.}

\section{Conclusion}

Mention multi-fidelity.

\section*{Acknowledgements}
MAF is supported by a National Science Foundation Graduate Research Fellowship under grant No. DGE-1326120.
MFH is supported by a National Aeronautics and Space Administration FINESST under grant No. ASTRO20-0022.
SB is supported by NSF grant AST-1817256.

Computing resources were provided by Frontera LRAC AST21005.
The authors acknowledge the Frontera computing project at the Texas Advanced Computing Center (TACC) for providing HPC and storage resources that have contributed to the research results reported within this paper.
Frontera is made possible by National Science Foundation award OAC-1818253.
URL: \url{http://www.tacc.utexas.edu}

\section*{Data Availability}
Flux power spectra generated from the low resolution, high resolution, and testing sets are available at.
Both HDF5 and plain text (appropriate for multi-fidelity emulation) formats are available.
Single- and multi-fidelity emulator predictions for the $10$ testing simulations are also available from the same repository.
The spectra underlying the flux power are available upon request.
\spb{We should try to make the spectra and flux power available for this one, but only after the likelihood paper is written.}

\bibliographystyle{JHEP}
\bibliography{refs}

\appendix

\label{lastpage}
\end{document}
