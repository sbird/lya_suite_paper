\section{Introduction}\label{sec:intro}

The \lya forest \citep{1965ApJ...142.1633G, 1998ApJ...495...44C, 1998MNRAS.301..478T, 2000ApJ...543....1M, 2001ApJ...552...15H, 2002MNRAS.329..848V, 2006AJ....132..117F, 2006MNRAS.365..231V, 2006ApJS..163...80M} measures the distribution of neutral gas at relatively low densities. This gas traces the growth of cosmic structure, making the \lya~forest an exceptionally powerful cosmological probe, sensitive to the distribution of dark matter deep in the matter dominated era. Correlating absorption from different quasar sightlines has allowed detection of the baryon acoustic oscillations and constraints on the expansion of the universe \cite{2011JCAP...09..001S, 2013JCAP...04..026S, 2020ApJ...901..153D, 2022arXiv220913942C}.
The densities probed by the \lya forest from redshift $z=2-5$ are $\sim 1-100 \ \times$ the cosmological mean density.
For these redshifts and densities stellar winds and star formation effects are negligible, though feedback from black holes can be important \citep{2013MNRAS.429.1734V, 2020MNRAS.495.1825C}. Thus the \lya~forest is also able to measure the primordial fluctuations on some of the smallest scales available, $k \sim 1$ h/Mpc \citep{2004MNRAS.354..684V, 2005ApJ...635..761M, 2006MNRAS.370L..51V, 2005PhRvD..71j3515S, 2006JCAP...10..014S, 2017JCAP...06..047Y, 2020JCAP...04..038P, 2021JCAP...03..049G}. In addition, the \lya~forest is sensitive to the thermal and ionization history of the intergalactic medium (IGM) \citep{2008MNRAS.386.1131B,2014MNRAS.438.2499B, 2016MNRAS.463.2335N,2019ApJ...872...13W, 2019ApJ...872..101B, 2019MNRAS.490.3177W,2021MNRAS.506.4389G, 2022ApJ...933...59V}, and by constraining the free-streaming length of small structures, the mass scale of thermal relic dark matter \citep{2005PhRvD..71f3534V,  2013PhRvD..88d3502V, 2017PhRvD..96b3522I, 2020JCAP...04..038P, 2021MNRAS.502.2356G, 2021PhRvL.126g1302R, 2022arXiv220914220V}.

%The \lya forest is a series of spectral features, produced by overlapping neutral hydrogen absorption profiles in the spectra of distant luminous quasars as their light is processed through neutral hydrogen in the intergalactic medium (IGM) \citep{1965ApJ...142.1633G}.
%In the rest frame of this neutral hydrogen, light that has been redshifted close to the Lyman-$\alpha$ transition at $1215.67${\AA} is absorbed.
%As the light traverses the IGM, it continues to intersect additional neutral hydrogen at lower redshift.
%The result is a quasar transmission spectra containing an overlapping field of absorption features, which measures the neutral gas density along that sightline \citep{1998ApJ...495...44C}.

The extended Baryon Oscillation Sky Survey (eBOSS), part of the Sloan Digital Sky Survey (SDSS) \cite{2019JCAP...07..017C}, has computed the 1D flux power spectrum along the line of sight to the quasar from over $43,000$ quasars, with a statistical error $\sim 1\%$ at some redshifts. This exceptional statistical error means that the error budget is dominated by systematic uncertainty, especially uncertainty in the resolution of the spectrograph on small scales \cite{2019JCAP...07..017C}. The Dark Energy Spectroscopic Instrument (DESI) has improved the spectrograph resolution by a factor of two \cite{2022AJ....164..207A}.
Thus, early data from DESI has measured the flux power spectrum at smaller scales ($k \gtrsim 0.035$ km$^{-1}$ s) than SDSS \cite{2023arXiv230606316G, 2023arXiv230606311R}. Future releases will measure higher redshifts ($z>4.6$) and increase the number of \lya forest quasar spectra by a factor of four over SDSS \cite{2016arXiv161100036D}.

Modelling the \lya~forest requires numerical simulations able to follow the distribution of gas on small scales. In this paper we present cosmological parameter inference using a new likelihood built on the PRIYA simulation suite \cite{2023simsuite}. The PRIYA simulations are in $120$ Mpc/h boxes, and are comprised of $48$ simulations with $2\times 1536^3$ particles (and thus a mean inter-particle spacing of $78$ kpc/h), as well as $3$ simulations with $2\times 3072^3$ particles (and thus a mean inter-particle spacing of $39$ kpc/h). The highest resolution exceeds the resolution of state-of-the-art galaxy formation simulations such as Illustris-TNG \cite{2018MNRAS.475..676S}. PRIYA is run with the same highly scalable MP-Gadget code as the ASTRID simulation \cite{2022MNRAS.512.3703B,2022MNRAS.513..670N}, and contains full hydrodynamic simulations with models of galaxy formation and black hole feedback to $z=2$. PRIYA is thus the first cosmological simulation suite which achieves, in a single box, the required box size of $120$ Mpc/h, capable of minimising sample variance in the \lya~forest\cite{2014JCAP...07..005B}, and a resolution high enough to include the gas Jeans' scale. Importantly, this removes the need for the `splicing' correction used in earlier work to combine different boxsizes into a single whole \cite{2014JCAP...07..005B,2020JCAP...04..038P}.

The PRIYA simulations are used to build multi-fidelity emulators  \cite{2019JCAP...02..050B, 2022MNRAS.509.2551H, 2022MNRAS.517.3200F} for the flux power spectrum and the mean temperature of the IGM. Each emulator is a surrogate model, able to reproduce the 1D flux power spectrum or mean IGM temperature for cosmological parameters (within the prior simulation volume) to $\sim 1 \%$ accuracy. A multi-fidelity emulator combines two different resolution training samples. Many low fidelity samples are used to explore parameter space, and their output is corrected with a few high fidelity samples. The combination is thus able to make predictions for the highest resolution simulation at a fraction of the computational cost of a single fidelity emulator \cite{10.1093/biomet/87.1.1, 2022MNRAS.509.2551H}. Emulators have been used to study various cosmological probes: the matter power spectrum \citep{Heitmann:2009, Heitmann:2014, Lawrence:2017, Giblin:2019, Euclid:2021, Arico:2021, Giri:2021}, weak lensing shear \citep{Harnois:2019, Davies:2021}, the halo mass function \citep{McClintock:2019, Nishimichi:2019, Bocquet:2022}, the 21-cm signal \citep{Kern:2017, Cohen:2020, Bevins:2021, Bye:2022} and the \lya forest \citep{2019JCAP...02..050B, Rogers:2019, 2021JCAP...05..033P, 2021JCAP...04..059W, Rogers:2021a,2021PhRvL.126g1302R}. Here, we present the first emulator for the eBOSS \lya~forest and the first cosmological constraints derived from it. Our multi-fidelity emulator is similar to that described in Ref.~\cite{2022MNRAS.517.3200F}, but the simulation volume has been increased by a factor of $64$, and the spatial resolution has been improved by a factor of $1.5$. We also use mean IGM temperature data \cite{2021MNRAS.506.4389G} to constrain the parameters of Helium reionization, data which is ultimately derived from higher resolution quasar surveys \citep{2017MNRAS.466.4332I, 2022MNRAS.509.2842K, 2019MNRAS.489.2536D}.

%\textbf{Our work has several novel features, including the use of full hydrodynamic simulations, the implementation of a physical model for hydrogen and helium reionization, and the use of a multi-fidelity emulator for sampling.}

In summary, our method is: (1) Construct an emulator for the 1D \lya~flux power spectrum using the PRIYA simulations, as described in Ref.~\cite{2023simsuite}, Section Section~\ref{sec:emulator} and Section~\ref{sec:simulations}. (2) Augment observational errors with estimates for the residual theoretical uncertainty to build a covariance matrix, and correct the flux power spectra for metal contamination as described in Section~\ref{sec:inference}. (3) Use this emulator and likelihood to do Markov Chain Monte Carlo and constrain cosmological parameters with results described in Section~\ref{sec:results}. We discuss some caveats and compare to earlier work in Section~\ref{sec:discussion} and our conclusions are presented in Section~\ref{sec:conclusions}.

%We begin, in Section~\ref{sec:emulator}, by describing Gaussian processes, and the emulator methods used for both a single- and multi-fidelity emulator model.
%We then provide details on the simulation suite used for training the emulator in Section~\ref{sec:simulations}, including the parameters that are varied, the method of sampling those parameters to construct the final suite, as well as the method of calculating the summary statistics used (flux power and mean temperature).
%In Section~\ref{sec:inference} we outline and discuss the inference scheme, the observational data that is used and the construction of the likelihood.
%Section~\ref{sec:results} details the results of running the likelihood in an MCMC framework, and we conclude in Section~\ref{sec:conclusions}.

%There are also appendices for extended and alternative results.
%a test where we use simulation input as the observation to see how well our likelihood recovers the truth in Appendix~\ref{sec:simdat};
%results from an analysis that includes priors on some of the parameters in Appendix~\ref{sec:priors};
%We describe the impact of adding interpolation error estimated by the Gaussian process in Appendix~\ref{sec:loovsgperr}. Appendix~\ref{sec:dr9_results} describes 
%results using an older observed flux power spectrum (DR9) in ;
%results where we have inflated the observational errors in Appendix~\ref{sec:xboss};
%the full set of parameter posteriors in Appendix~\ref{sec:full_posteriors};
% results using a single-fidelity emulator in Appendix~\ref{sec:sf_results};
%results using only the mean temperature for inference in Appendix~\ref{sec:t0-only}.
%and results using a reduced range for the scales in Appendices~\ref{sec:reducedk}.

MCMC chains for all the results presented in this work along with files containing the training outputs used to construct the emulators\footnote{\url{https://github.com/mafern/InferenceLyaData}}, as well as the code\footnote{\url{https://github.com/sbird/lya_emulator_full}}, which includes the emulator, likelihood, and integration with the Cobaya MCMC package, are available publicly.

%\spb{Material taken out of the above}
%Modern cosmological inference requires a constantly evolving set of tools, from improved modeling of cosmology and astrophysics, to new computational tools to keep pace with modern observations.
%In particular, the march of modern observations towards smaller scales and larger samples require more computationally intensive modeling, i.e. running larger volume and higher resolution simulations.
%The need for these costly simulations introduces its own problem; fewer of them can be run, reducing their usefulness in inference frameworks.
%Various methods have been developed to answer this last problem, many of which use machine learning methods.

%Parameter inference tasks, such as Markov Chain Monte Carlo analysis, can require up to $\sim10^6$ model evaluations.
%This is prohibitive if each evaluation requires an expensive cosmological simulation, but easily achievable by drawing from the emulator based surrogate.

%Here, we use a Gaussian process (GP) based emulator following \cite{2019JCAP...02..050B, 2022MNRAS.517.3200F}.
%Gaussian processes are flexible, fast and easy to train, producing good results even for our relatively small training data set.
%Gaussian processes \citep{2006gpml.book.....R} are a means of interpolating between the simulation outputs via a distribution of functions that is learned through training on simulations.
%Draws from the ($n$-dimensional) learned distribution are potential predictions, with the mean serving as the actual prediction (best estimate).
%The prediction uncertainty comes from the variance of the learned distribution.
%The predictions can be made for arbitrary simulation inputs, within the parameter limits of the training set.

%This model was used to emulate the matter power spectrum in \cite{2022MNRAS.509.2551H}, and more recently we used this model for the \lya forest flux power spectrum \cite{2022MNRAS.517.3200F}.
%In \cite{2022MNRAS.517.3200F}, the training simulations were split into two fidelities; a sample of $40$ low resolution simulations (LF), and a subset of $6$ of the LF simulations run at higher resolution (HF).
%\textbf{The multi-fidelity emulator used in this work follows the same division, a small sample of HF simulations, supported by a large sample of much cheaper LF simulations.}
