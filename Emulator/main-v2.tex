\documentclass[a4paper,11pt]{article}
% \pdfoutput=1
\usepackage{jcappub}
% \usepackage[T1]{fontenc} % if needed
\usepackage{amsmath}
\usepackage{amssymb}

\DeclareRobustCommand{\ion}[2]{%
\relax\ifmmode
\ifx\testbx\f@series
{\mathbf{#1\,\mathsc{#2}}}\else
{\mathrm{#1\,\mathsc{#2}}}\fi
\else\textup{#1\,{\mdseries\textsc{#2}}}%
\fi}

\usepackage{enumitem}
\setlist[enumerate]{itemsep=1mm}

\newcommand{\lya}{Lyman-$\alpha$~}
\newcommand{\Lya}{Lyman-$\alpha$~}
% martin comment function
\newcommand{\maf}[1]{{\textcolor{red}{[{\bf MAF}: #1]}}}
% simeon comment function
\newcommand{\spb}[1]{{\textcolor{magenta}{[{\bf SPB}: #1]}}}
% ming-feng comment function
\newcommand{\mfho}[1]{{\textcolor{green}{[{\bf MFH}: #1]}}}

\newcommand{\apjs}{ApJ Supplement}
\newcommand{\apj}{ApJ}
\newcommand{\apjl}{ApJL}

\newcommand{\aap}{AAP}
\newcommand{\mnras}{MNRAS}
\newcommand{\prd}{PRD}
\newcommand{\prl}{PRL}
\newcommand{\jcap}{JCAP}

%\title{\boldmath Inference via a \lya Forest Emulator}
\title{\boldmath Cosmological Constraints from the eBOSS \lya Forest using the PRIYA Simulations}

\author{M.A. Fernandez,}\emailAdd{mfern027@ucr.edu}
\author{Simeon Bird}\emailAdd{sbird@ucr.edu}
\author{and Ming-Feng Ho,}\emailAdd{mho026@ucr.edu}
\affiliation{Department of Physics and Astronomy, University of California Riverside, 900 University Ave, Riverside, CA 92521}

\abstract{
We present new cosmological parameter constraints from the eBOSS \lya forest survey. We use a new theoretical model and likelihood based on the PRIYA simulation suite. PRIYA is the first suite to resolve the \lya forest in a ($120$~Mpc/h~)$^3$ volume, using a multi-fidelity emulation technique. We use PRIYA to predict \lya forest observables with $\lesssim 1\%$ interpolation error over an $11$ dimensional ($9$ simulated, $2$ in post-processing) parameter space. We identify an internal tension within the flux power spectrum data. 
Once the discrepant data is removed, we find the primeval scalar spectral index measured at a pivot scale of $k_0 = 0.78$ Mpc$^{-1}$ to be $n_P = 1.009^{+0.027}_{-0.018}$ at 68\% confidence. This measurement from the \lya forest flux power spectrum alone is in \textbf{reasonable} agreement with Planck, and in tension with earlier eBOSS analyses. The amplitude of matter fluctuations is $\sigma_8 = 0.733^{+0.026}_{-0.029}$ at 68\% confidence, in agreement with Dark Energy Survey weak lensing measurements and other small-scale structure probes and in tension with CMB measurements from Planck and ACT. The effective optical depth to \lya photons from our pipeline is in good agreement with earlier high resolution measurements. \textbf{We find a linear power at $z=3$ and $k = 0.009$ s/km of $\Delta_L^2 = 0.302^{+0.024}_{-0.027}$ with a slope $n_\mathrm{eff} = -2.264^{+0.026}_{-0.018}$. Our flux power spectrum only chains prefer a low level of heating during helium reionization. When we add IGM temperature data we find $n_P = 0.983\pm 0.020$ and $\sigma_8 = 0.703^{+0.023}_{-0.027}$.
Our chains prefer an early and long helium reionization event, as suggested by measurements from the helium \lya forest.}
In the near future we will use our pipeline to infer cosmological parameters from the DESI \lya data.
}


\begin{document}
\maketitle
\flushbottom
% - Different seed cosmic variance chain added to section 3.3
% - Plot like loo_vs_emu_error_wlegend.pdf but for z=2.2 - 2.6 and use it as main result, rather than the shrunk corner plots.
% - Drop LOO from the 'full posteriors' figure. Add 'with mean temperature'? Or just leave it alone?
%
% Main conclusion:
% There is a discrepancy between the z=2.2-2.4 and z >= 2.6 bins. This drives tensions across the whole parameter space. Clear from the best fit.

\section{Introduction}\label{sec:intro}

The \lya forest \citep{1965ApJ...142.1633G, 1998ApJ...495...44C, 1998MNRAS.301..478T, 2000ApJ...543....1M, 2001ApJ...552...15H, 2002MNRAS.329..848V, 2006AJ....132..117F, 2006MNRAS.365..231V, 2006ApJS..163...80M} measures the distribution of neutral gas at relatively low densities. This gas traces the growth of cosmic structure, making the \lya~forest an exceptionally powerful cosmological probe, sensitive to the distribution of dark matter deep in the matter dominated era. Correlating absorption from different quasar sightlines has allowed detection of the baryon acoustic oscillations and constraints on the expansion of the universe \cite{2011JCAP...09..001S, 2013JCAP...04..026S, 2020ApJ...901..153D, 2022arXiv220913942C}.
The densities probed by the \lya forest from redshift $z=2-5$ are $\sim 1-100 \ \times$ the cosmological mean density.
For these redshifts and densities stellar winds and star formation effects are negligible, though feedback from black holes can be important \citep{2013MNRAS.429.1734V, 2020MNRAS.495.1825C}. Thus the \lya~forest is also able to measure the primordial fluctuations on some of the smallest scales available, $k \sim 1$ h/Mpc \citep{2004MNRAS.354..684V, 2005ApJ...635..761M, 2006MNRAS.370L..51V, 2005PhRvD..71j3515S, 2006JCAP...10..014S, 2017JCAP...06..047Y, 2020JCAP...04..038P, 2021JCAP...03..049G}. In addition, the \lya~forest is sensitive to the thermal and ionization history of the intergalactic medium (IGM) \citep{2008MNRAS.386.1131B,2014MNRAS.438.2499B, 2016MNRAS.463.2335N,2019ApJ...872...13W, 2019ApJ...872..101B, 2019MNRAS.490.3177W,2021MNRAS.506.4389G, 2022ApJ...933...59V}, and by constraining the free-streaming length of small structures, the mass scale of thermal relic dark matter \citep{2005PhRvD..71f3534V,  2013PhRvD..88d3502V, 2017PhRvD..96b3522I, 2020JCAP...04..038P, 2021MNRAS.502.2356G, 2021PhRvL.126g1302R, 2022arXiv220914220V}.

%The \lya forest is a series of spectral features, produced by overlapping neutral hydrogen absorption profiles in the spectra of distant luminous quasars as their light is processed through neutral hydrogen in the intergalactic medium (IGM) \citep{1965ApJ...142.1633G}.
%In the rest frame of this neutral hydrogen, light that has been redshifted close to the Lyman-$\alpha$ transition at $1215.67${\AA} is absorbed.
%As the light traverses the IGM, it continues to intersect additional neutral hydrogen at lower redshift.
%The result is a quasar transmission spectra containing an overlapping field of absorption features, which measures the neutral gas density along that sightline \citep{1998ApJ...495...44C}.

The extended Baryon Oscillation Sky Survey (eBOSS), part of the Sloan Digital Sky Survey (SDSS) \cite{2019JCAP...07..017C}, has computed the 1D flux power spectrum along the line of sight to the quasar from over $43,000$ quasars, with a statistical error $\sim 1\%$ at some redshifts. This exceptional statistical error means that the error budget is dominated by systematic uncertainty, especially uncertainty in the resolution of the spectrograph on small scales \cite{2019JCAP...07..017C}. The Dark Energy Spectroscopic Instrument (DESI) has improved the spectrograph resolution by a factor of two \cite{2022AJ....164..207A}.
Thus, early data from DESI has measured the flux power spectrum at smaller scales ($k \gtrsim 0.035$ km$^{-1}$ s) than SDSS \cite{2023arXiv230606316G, 2023arXiv230606311R}. Future releases will measure higher redshifts ($z>4.6$) and increase the number of \lya forest quasar spectra by a factor of four over SDSS \cite{2016arXiv161100036D}.

Modelling the \lya~forest requires numerical simulations able to follow the distribution of gas on small scales. In this paper we present cosmological parameter inference using a new likelihood built on the PRIYA simulation suite \cite{2023simsuite}. The PRIYA simulations are in $120$ Mpc/h boxes, and are comprised of $48$ simulations with $2\times 1536^3$ particles (and thus a mean inter-particle spacing of $78$ kpc/h), as well as $3$ simulations with $2\times 3072^3$ particles (and thus a mean inter-particle spacing of $39$ kpc/h). The highest resolution exceeds the resolution of state-of-the-art galaxy formation simulations such as Illustris-TNG \cite{2018MNRAS.475..676S}. PRIYA is run with the same highly scalable MP-Gadget code as the ASTRID simulation \cite{2022MNRAS.512.3703B,2022MNRAS.513..670N}, and contains full hydrodynamic simulations with models of galaxy formation and black hole feedback to $z=2$. PRIYA is thus the first cosmological simulation suite which achieves, in a single box, the required box size of $120$ Mpc/h, capable of minimising sample variance in the \lya~forest\cite{2014JCAP...07..005B}, and a resolution high enough to include the gas Jeans' scale. Importantly, this removes the need for the `splicing' correction used in earlier work to combine different boxsizes into a single whole \cite{2014JCAP...07..005B,2020JCAP...04..038P}.

The PRIYA simulations are used to build multi-fidelity emulators  \cite{2019JCAP...02..050B, 2022MNRAS.509.2551H, 2022MNRAS.517.3200F} for the flux power spectrum and the mean temperature of the IGM. Each emulator is a surrogate model, able to reproduce the 1D flux power spectrum or mean IGM temperature for cosmological parameters (within the prior simulation volume) to $\sim 1 \%$ accuracy. A multi-fidelity emulator combines two different resolution training samples. Many low fidelity samples are used to explore parameter space, and their output is corrected with a few high fidelity samples. The combination is thus able to make predictions for the highest resolution simulation at a fraction of the computational cost of a single fidelity emulator \cite{10.1093/biomet/87.1.1, 2022MNRAS.509.2551H}. Emulators have been used to study various cosmological probes: the matter power spectrum \citep{Heitmann:2009, Heitmann:2014, Lawrence:2017, Giblin:2019, Euclid:2021, Arico:2021, Giri:2021}, weak lensing shear \citep{Harnois:2019, Davies:2021}, the halo mass function \citep{McClintock:2019, Nishimichi:2019, Bocquet:2022}, the 21-cm signal \citep{Kern:2017, Cohen:2020, Bevins:2021, Bye:2022} and the \lya forest \citep{2019JCAP...02..050B, Rogers:2019, 2021JCAP...05..033P, 2021JCAP...04..059W, Rogers:2021a,2021PhRvL.126g1302R}. Here, we present the first emulator for the eBOSS \lya~forest and the first cosmological constraints derived from it. Our multi-fidelity emulator is similar to that described in Ref.~\cite{2022MNRAS.517.3200F}, but the simulation volume has been increased by a factor of $64$, and the spatial resolution has been improved by a factor of $1.5$. We also use mean IGM temperature data \cite{2021MNRAS.506.4389G} to constrain the parameters of Helium reionization, data which is ultimately derived from higher resolution quasar surveys \citep{2017MNRAS.466.4332I, 2022MNRAS.509.2842K, 2019MNRAS.489.2536D}.

%\textbf{Our work has several novel features, including the use of full hydrodynamic simulations, the implementation of a physical model for hydrogen and helium reionization, and the use of a multi-fidelity emulator for sampling.}

In summary, our method is: (1) Construct an emulator for the 1D \lya~flux power spectrum using the PRIYA simulations, as described in Ref.~\cite{2023simsuite}, Section Section~\ref{sec:emulator} and Section~\ref{sec:simulations}. (2) Augment observational errors with estimates for the residual theoretical uncertainty to build a covariance matrix, and correct the flux power spectra for metal contamination as described in Section~\ref{sec:inference}. (3) Use this emulator and likelihood to do Markov Chain Monte Carlo and constrain cosmological parameters with results described in Section~\ref{sec:results}. We discuss some caveats and compare to earlier work in Section~\ref{sec:discussion} and our conclusions are presented in Section~\ref{sec:conclusions}.

%We begin, in Section~\ref{sec:emulator}, by describing Gaussian processes, and the emulator methods used for both a single- and multi-fidelity emulator model.
%We then provide details on the simulation suite used for training the emulator in Section~\ref{sec:simulations}, including the parameters that are varied, the method of sampling those parameters to construct the final suite, as well as the method of calculating the summary statistics used (flux power and mean temperature).
%In Section~\ref{sec:inference} we outline and discuss the inference scheme, the observational data that is used and the construction of the likelihood.
%Section~\ref{sec:results} details the results of running the likelihood in an MCMC framework, and we conclude in Section~\ref{sec:conclusions}.

%There are also appendices for extended and alternative results.
%a test where we use simulation input as the observation to see how well our likelihood recovers the truth in Appendix~\ref{sec:simdat};
%results from an analysis that includes priors on some of the parameters in Appendix~\ref{sec:priors};
%We describe the impact of adding interpolation error estimated by the Gaussian process in Appendix~\ref{sec:loovsgperr}. Appendix~\ref{sec:dr9_results} describes 
%results using an older observed flux power spectrum (DR9) in ;
%results where we have inflated the observational errors in Appendix~\ref{sec:xboss};
%the full set of parameter posteriors in Appendix~\ref{sec:full_posteriors};
% results using a single-fidelity emulator in Appendix~\ref{sec:sf_results};
%results using only the mean temperature for inference in Appendix~\ref{sec:t0-only}.
%and results using a reduced range for the scales in Appendices~\ref{sec:reducedk}.

MCMC chains for all the results presented in this work along with files containing the training outputs used to construct the emulators\footnote{\url{https://github.com/mafern/InferenceLyaData}}, as well as the code\footnote{\url{https://github.com/sbird/lya_emulator_full}}, which includes the emulator, likelihood, and integration with the Cobaya MCMC package, are available publicly.

%\spb{Material taken out of the above}
%Modern cosmological inference requires a constantly evolving set of tools, from improved modeling of cosmology and astrophysics, to new computational tools to keep pace with modern observations.
%In particular, the march of modern observations towards smaller scales and larger samples require more computationally intensive modeling, i.e. running larger volume and higher resolution simulations.
%The need for these costly simulations introduces its own problem; fewer of them can be run, reducing their usefulness in inference frameworks.
%Various methods have been developed to answer this last problem, many of which use machine learning methods.

%Parameter inference tasks, such as Markov Chain Monte Carlo analysis, can require up to $\sim10^6$ model evaluations.
%This is prohibitive if each evaluation requires an expensive cosmological simulation, but easily achievable by drawing from the emulator based surrogate.

%Here, we use a Gaussian process (GP) based emulator following \cite{2019JCAP...02..050B, 2022MNRAS.517.3200F}.
%Gaussian processes are flexible, fast and easy to train, producing good results even for our relatively small training data set.
%Gaussian processes \citep{2006gpml.book.....R} are a means of interpolating between the simulation outputs via a distribution of functions that is learned through training on simulations.
%Draws from the ($n$-dimensional) learned distribution are potential predictions, with the mean serving as the actual prediction (best estimate).
%The prediction uncertainty comes from the variance of the learned distribution.
%The predictions can be made for arbitrary simulation inputs, within the parameter limits of the training set.

%This model was used to emulate the matter power spectrum in \cite{2022MNRAS.509.2551H}, and more recently we used this model for the \lya forest flux power spectrum \cite{2022MNRAS.517.3200F}.
%In \cite{2022MNRAS.517.3200F}, the training simulations were split into two fidelities; a sample of $40$ low resolution simulations (LF), and a subset of $6$ of the LF simulations run at higher resolution (HF).
%\textbf{The multi-fidelity emulator used in this work follows the same division, a small sample of HF simulations, supported by a large sample of much cheaper LF simulations.}

\section{Simulation Suite and Emulator}
\label{sec:emulator}
\label{sec:simulations}

\begin{table}
	\centering
     \begin{tabular}{|c|c|c|c|c|}
		\hline
		Simulation & Box Volume & N$_{\text{gas}}$ & M$_{\text{gas}}$ (M$_{\odot}$ h$^{-1}$)\\
		\hline
		LF & $(120$ Mpc h$^{-1})^3$ & $1536^3$ & $[5.29, 6.98]\times10^6$\\
		HF & $(120$ Mpc h$^{-1})^3$ & $3072^3$ & $[6.73, 7.97]\times10^5$\\
		\hline
	\end{tabular}
    \caption{\label{table:simulations}
    Low-Fidelity (LF) and High-Fidelity (HF) simulation suite details.
    N$_{\text{gas}}$ is the number of gas particles simulated, M$_{\text{gas}}$ is the resulting mass resolution of those particles.}
\end{table}

In this Section, we briefly describe the properties of the simulations and emulator, and refer the reader to Ref.~\cite{2023simsuite} for the full details.
%We use the multi-fidelity Gaussian process (GP) based emulator, using the simulation suite validated and constructed in Ref.~\cite{2023simsuite}. 
The emulator allows predictions for the output of a simulation at an arbitrary set of cosmological parameters within our prior volume with an average interpolation error of $0.2\%$ at low fidelity and $1\%$ at high fidelity.
%The emulator also includes a measure of prediction uncertainty. 
Our multi-fidelity emulator combines simulations at different resolutions, following the scheme outlined in Ref.~\cite{2022MNRAS.517.3200F}. The emulator combines low fidelity (LF) and high fidelity (HF) simulations. Box volume, %(simulations use periodic boundaries), 
number of gas particles, and gas particle mass resolution are reported in Table~\ref{table:simulations}. We performed a total of $48$ low fidelity (LF) and $3$ high fidelity (HF) simulations. Low fidelity simulations have  $1536^3$ particles, while high fidelity simulations have $3072^3$ particles. Sampled parameters are chosen to maximise spread in parameter space, as described fully in Ref.~\cite{2023simsuite}.

\begin{figure}
    \centering
    \includegraphics[width=\columnwidth]{figures/spectra_temp_simulation.png}
    \caption{\label{fig:spec_sim}
    Example Lyman-$\alpha$ forest spectra and corresponding gas density and temperature (colors) from an LF and HF simulation at redshift $z=4$.
    The top panel shows high resolution and the bottom panel shows low resolution. The middle panel shows the \lya forest spectra for the skewers passing through the middle of the top panel (high resolution, yellow line) and bottom panel (low resolution, red line).
    }
\end{figure}
The range given for the gas mass resolution is due to the varying value of $h$ in our simulation suite ($\Omega_b h^2$ is fixed, at a value of $0.0224$). We show in Ref.~\cite{2023simsuite} that this gas mass is sufficient for the scales and redshifts probed by the eBOSS flux power spectrum. Our simulations include a full galaxy physics model with star formation, stellar and AGN feedback and inhomogeneous reionization models. Simulations were performed using MP-Gadget\footnotemark, an N-body and smoothed particle hydrodynamics (SPH) code.\footnotetext{\url{https://github.com/MP-Gadget/MP-Gadget}}
MP-Gadget uses the gravitational timestepping algorithm from Gadget-4 \cite{Springel:2021}, and various other algorithmic improvements \cite{2020JCAP...06..002B}.
Simulations are initialised at $z=99$ and finish at $z=2.2$. The galaxy formation model is similar to the \texttt{ASTRID} simulation \cite{2022MNRAS.512.3703B, 2022MNRAS.513..670N} and is described fully in Ref.~\cite{2023simsuite}.

%In \cite{2009MNRAS.398L..26B}, the recommended gas mass resolution for the \lya forest is $2\times10^5$ M$_{\odot}$ h$^{-1}$ at $z=5$ and $1.61\times10^6$ M$_{\odot}$ h$^{-1}$ at $z=2$.
%Our HF simulations have better gas mass resolution than the recommendation for lower redshifts, and slightly worse resolution than recommended at higher redshifts.
%\textbf{However, we include a temperature boost \cite{2019ApJ...874..154D} due to impulsive heating during hydrogen reionization from ionization fronts.
%This increase the thermal free-streaming length at high redshift, which reduces the resolution requirements.}
%Also, we currently restrict our analysis to $z=2.2$ to $z=4.6$, so the recommendation at $z=5$ is more strict than required in this work.

%In \cite{2014JCAP...07..005B}, simulations with box side length of at least $100$ Mpc h$^{-1}$, and a corresponding gas mass resolution of $4.3\times10^5 M_\odot/h$ were recommended for the \lya forest.
%\textbf{Our HF simulations are larger volume than the above recommendation, with slightly lower gas mass resolution.
%In \cite{2014JCAP...07..005B}, they found that simulations run with a similar resolution to our HF samples were no worse than $5\%$ converged.
%When the reduced resolution requirement due to the aforementioned temperature boost is considered, the simulations used in this work simultaneously achieve the volume and resolution recommended in \cite{2014JCAP...07..005B}, without the need for splicing.
%}

\subsection{Cosmological \& Astrophysical Parameters}\label{sec:parameters}

Table~\ref{tab:emulatorparams} summarises the parameters that are varied across our suite of simulations, as well as their limits.
We model the primordial power spectrum $P(k)$ using two parameters: a slope, $n_P$, and an amplitude, $A_P$:
\begin{equation}
    P(k) = A_P \left(\frac{k}{0.78 \mathrm{h/Mpc}}\right)^{n_P - 1}\,.
\end{equation}
We also vary the Hubble parameter $h$, and the total matter density through $\Omega_M h^2$, although we will see these are not strongly constrained by the \Lya~forest. We add three parameters for the He~{\sc ii} reionization model \cite{2020MNRAS.496.4372U}: $z_{Hei}$ and $z_{Hef}$ are the redshifts for the start and end of He~{\sc ii} reionization, and $\alpha_q$ is the quasar spectral index (which scales the peak temperature during He~{\sc ii} reionization).
$z_{Hi}$ is the midpoint redshift of H~{\sc i} reionization.
Finally, $\epsilon_{AGN}$ is the black hole feedback factor, to which the \lya~forest is insensitive.

\begin{table*}
\begin{centering}
  \begin{tabular}{llll}
  \hline
  Parameter & Minimum & Maximum & Description \\
    \hline
    $n_P$  &  $0.8$  & $0.995$ & Scalar spectral index \\
    $A_P$  &  $1.2 \times 10^{-9}$  & $2.6 \times 10^{-9}$ & Power amplitude at $k = 0.78$ h/Mpc \\
    $h$    & $0.65$  & $0.75$ & Hubble parameter \\
    $\Omega_0 h^2$ & $0.14$ & $0.146$ & Total matter density \\
    $z_{Hei}$      & $3.5$  & $4.1$  & Start redshift of HeII reionization \\
    $z_{Hef}$      & $2.6$  & $3.2$  & End redshift of HeII reionization \\
    $\alpha_q$     & $1.3$  & $2.5$ & Quasar spectral index during HeII reionization  \\
    $z_{Hi}$        & $6.5$ & $8$   & Median redshift of HI reionization \\
    $\epsilon_{AGN}$ & $0.03$ & $0.07$ & Thermal efficiency of black hole feedback \\
    $\tau_0$ & $0.75$ & $1.25$ & Mean optical depth at $z=3$ in Eq.~\ref{eq:meanflux}.\\
    $d \tau_0$ & $-0.4$ & $0.25$ & Mean optical depth redshift evolution in Eq.~\ref{eq:meanflux}. \\
    \hline
  \end{tabular}
  \caption{Summary of likelihood function parameters, together with the ranges covered by the emulator. We vary a total of $11$ parameters: $4$ for cosmology, $3$ for the helium reionization model, $1$ for the hydrogen reionization model, $1$ for the strength of AGN feedback and $2$ for the mean optical depth.}
  \label{tab:emulatorparams}
  \end{centering}
\end{table*}

There are two further parameters for the \Lya~effective optical depth, varied by post-processing the artificial spectra. We parametrise the mean flux $\mathcal{F} = \exp(-\tau)$ around the power law redshift evolution from Ref.~\cite{2007MNRAS.382.1657K}, as
\begin{equation}
\tau^{\text{eff}}_{\text{H~{\sc i}}} = 0.0023 \times \tau_0 \times (1+z)^{3.65 + d\tau_0}\,.
 \label{eq:meanflux}
\end{equation}
The parameters varied are $\tau_0$ and $d\tau_0$, with $(1, 0)$ corresponding to the redshift evolution of Ref.~\cite{2007MNRAS.382.1657K}.
As the mean flux is chosen in post-processing, we can dramatically over-sample these parameters. The final set of flux power spectra is thus ten times the number of simulations, or $480$ LF, and $30$ HF simulated flux power spectra. We have freedom to vary the \Lya~mean flux as it is degenerate with the amplitude of the UVB. 

%Note that our emulator construction is designed to easily accommodate more general evolution. Each set of simulated spectra is post-processed at each redshift into ten flux power spectra with a mean flux spread uniformly over a range chosen to include the mean fluxes .

% --------------------------------------------------------------------------------------------------

\subsection{Summary Statistics: Flux Power and IGM Temperature}\label{sec:sim_fps}

%In the companion paper to this work \cite{2023simsuite}, the \lya forest statistics extracted from these simulations are well converged . . .
Figure~\ref{fig:spec_sim} shows an example of the gas density and temperature (colors) at $z=4$ for both high and low resolution in our simulations, demonstrating how spectra connect to the matter density field.
We generate a total of $3\times 480^2 = 691,200$ spectra from each snapshot of each simulation, from $z=4.6$ to $z=2.2$ in increments of $\Delta z=0.2$, with a pixel resolution of $10$ km s$^{-1}$.
We generate \lya forest absorption spectra using Fake Spectra Flux Extractor \cite{2017ascl.soft10012B}\footnotemark, described in \cite{2015MNRAS.447.1834B}.
\footnotetext{\url{https://github.com/sbird/fake_spectra}}
We compute the 1D flux power spectrum of the \Lya~forest flux, averaged over all sightlines. The flux power is defined as 
\begin{equation}
 P_F(k) = |L^{-1}\tilde{\delta}^2_F(k)|\,.   
\end{equation}
$\tilde{\delta}^2_F(k)$ is the Fourier transform of the flux excess, $\delta_F(k) = F(k)/\langle F(k) \rangle - 1$, and $L$ is the length of the sightline. 

Our simulations contain a realistic population of DLAs, which we mask as in the observational pipeline. We extract the IGM mean temperatures directly from the simulation snapshots. First, the temperature and density for all the gas particles in the simulation are retrieved, then all particles that are within $5\%$ of the critical density are retained. The median temperature of these retained particles is used as the IGM mean temperature. All of the \lya forest flux power spectra, IGM mean temperatures, trained emulators, as well as select MCMC chains are available at: \url{https://github.com/mafern/InferenceLyaData}.

%
% Figure~\ref{fig:sim_fps} shows flux power spectra from all the LF (blue, solid) and HF (gold, dashed) simulations at several redshifts.
% Shown below each redshift panel in red are the ratios of the flux power spectra for simulations that were run at both resolutions.
% Note that there are ten times more flux power spectra shown here than simulations, due to the previously discussed post-processing method of mean flux rescaling.
%
% \begin{figure}
%     \centering
%     \includegraphics[width=\textwidth]{figures/fluxpower_comp.pdf}
%     \caption{\label{fig:sim_fps}
%     Flux power spectra from all LF (blue, solid) and HF (gold, dashed) simulations at $z=2.2$, $z=3.0$, $z=3.8$, and $z=4.6$.
%     In each panel below the flux power, the ratio of the flux power spectra for simulations run at both resolutions is shown, $LF/HF$.}
% \end{figure}

% --------------------------------------------------------------------------------------------------


% \begin{figure}
%     \centering
%     \includegraphics[width=0.66\textwidth]{figures/meant_range.pdf}
%     \caption{\label{fig:sim_temp}
%     IGM mean temperatures from all LF (blud, solid) and HF (gold, dashed) simulations for the full redshift range.
%     The bottom panel shows the ratio of the mean temperatures from simulations that were run at both resolutions, $LF/HF$.}
% \end{figure}

%--------------------------------------------------------------------------------------------------
% --------------------------------------------------------------------------------------------------
% --------------------------------------------------------------------------------------------------
% --------------------------------------------------------------------------------------------------
% --------------------------------------------------------------------------------------------------

\subsection{Gaussian Process Emulators}\label{sec:gps}

We use the Gaussian Process emulators described in Refs.~\cite{2022MNRAS.517.3200F, 2023simsuite} for the 1D flux power spectra and mean IGM temperature extracted from our simulations. The emulators interpolate over simulation outputs and make predictions for arbitrary parameter sets within the parameter limits shown in Table~\ref{tab:emulatorparams}. We use a multi-fidelity model, which allows simulations with different particle loads, and thus costs, to be combined together.
%This allows construction of an emulator that is a good approximation to a high fidelity model, but with most of the parameter space covered by cheaper, low fidelity simulations.
%makes full use of the range of scales and redshifts probed by \lya forest observations, at a significantly reduced computational cost.
%, as  (e.g. each HF simulation requires approximately sixteen times more node-hours to run than an LF simulation).
Specifically, we combine simulations run at two different resolutions, high fidelity (HF) and low fidelity (LF) specified in Table~\ref{table:simulations}. The multi-fidelity prediction for the HF outputs (shown here for the \lya forest flux power spectrum, but equally valid for the mean temperature) is given by a linear multi-fidelity model:
\begin{equation}
    P_F^{^\mathrm{HF}}(k, \boldsymbol{\theta}) = \rho(k) \cdot P_F^{^\mathrm{LF}}(k, \boldsymbol{\theta}) + \delta(k, \boldsymbol{\theta}),
    \label{eq:ko_model}
\end{equation}
where $\rho$ is a cosmology-independent but scale-dependent parameter, and $\delta(k, \boldsymbol{\theta})$ is a GP (independent of the LF output), both of which are optimized using the training samples.
%Since $\rho(k)$ is a multiplicative correction, and $\delta(k, \boldsymbol{\theta})$ is an additive correction,
% This is often called a linear multi-fidelity model.
We implement our multi-fidelity models using Emukit \cite{2021arXiv211013293P}.

Ref.~\cite{2023simsuite} quantifies the accuracy of our emulator using a leave-one-out technique, in which one simulation is chosen as a `left out' sample. A smaller emulator built excluding all samples from this simulation is used to predict the summary statistic for the left out sample. After computing leave-one-out interpolation errors for all potential test samples, we found on average $0.2\%$ accuracy for the low-fidelity simulations and $1\%$ for the high fidelity simulations. The last is likely a significant over-estimate of the actual error since the leave-one-out procedure in this case is missing $1/3$ of the total training data. The emulator accuracy indicates that the emulators are performing well, with potential errors significantly smaller than the 7\% average uncertainty in mean IGM temperature measurements \cite{2019JCAP...07..017C}.
Our likelihood function and emulator code is publicly available\footnote{\url{https://github.com/sbird/lya_emulator_full}}.

%Gaussian processes \citep{2006gpml.book.....R} are a trainable method for providing function prediction in a Bayesian framework.
%In this case, the GP is used to interpolate simulation outputs as a function of the simulation input parameters that are varied in the training samples (see Figure~\ref{fig:samples}).
%In a Gaussian process, a distribution of functions is learned through the training samples (simulation input/output pairs), and the mean of the learned distribution serves as the prediction, while the variance of the distribution quantifies the prediction uncertainty.
%\textbf{This allows a prediction to be cheaply produced for arbitrary simulation inputs (within the parameter space of the training samples -- GPs work poorly for extrapolation).}
%We use the GPy \cite{gpy2014} package to implement Gaussian processes within our emulator model.
%
% If $P_F(\boldsymbol{\theta})$ is the simulated \lya forest flux power spectrum (the following also applies to the IGM mean temperature, $T(\boldsymbol{\theta})$) as a function of $\boldsymbol{\theta}$, the input parameters to the simulation, then we are modeling this output as draws from a learned distribution,
% \begin{equation}
%     P_F(\boldsymbol{\theta}) \sim GP(\mu(\boldsymbol{\theta}), k(\boldsymbol{\theta}, \boldsymbol{\theta}^{\prime})),
% \end{equation}
% where $\mu(\boldsymbol{\theta})$ and $k(\boldsymbol{\theta}, \boldsymbol{\theta}^{\prime})$ are the mean and covariance function for the distribution, respectively.
% As in \cite{2022MNRAS.517.3200F}, we rescale the training samples by their median (the LF median for both single- and multi-fidelity emulators) so that we may assume a zero mean function during the GP training.
% The predictions are then multiplied by this rescaling factor to produce sensible predictions.
% With a zero mean function, it only remains for the covariance function to be trained, which amounts to selecting a covariance kernel and training the free parameters of that kernel.
% Here, as in \cite{2022MNRAS.517.3200F}, we use the sum of a radial basis function (RBF) kernel and a linear kernel to model our outputs.
%
% The total kernel (RBF + linear kernels) is given by:
% \begin{equation}
%         k_\mathrm{RBF}(\boldsymbol{\theta}, \boldsymbol{\theta}'; \sigma_0, \boldsymbol{l}) + k_\mathrm{LIN}(\boldsymbol{\theta}, \boldsymbol{\theta}'; \boldsymbol{\sigma})\\
%         = \sigma_0^2 \exp{\left( \sum_{i=1}^{d} -\frac{(\boldsymbol{\theta}_i - {\boldsymbol{\theta}_i}')^2}{2 l_i^2} \right)} +  \sum_{i=1}^{d} \sigma_i^2 \boldsymbol{\theta}_i {\boldsymbol{\theta}_i}',
% \end{equation}
% where $d$ is the dimensionality of the input parameters.
% The hyperparameters that are learned from the training samples are: variances for the RBF ($\sigma_0^2$) and  linear ($\boldsymbol{\sigma}^2$) kernels, and the lengthscale controlling the smoothness of the RBF kernel, $\boldsymbol{l}$.
% The RBF lengthscale and linear variance hyperparameters are shown as vectors here, indicating that an independent value is assigned for each dimension $d$ of the input for each of these hyperparameters.
% This is called Automatic Relevance Determination (ARD), and we use this in all cases for the LF \lya forest flux power spectrum.
% Including ARD allows the GP to train a separate hyperparameter for each dimension of the input, such that dimensions that have a more sensitive effect on the prediction can be given a smaller scale, e.g. the effect of $A_p$ on the mean temperature is negligible, while $\alpha_q$ has a large effect.
% For the IGM mean temperature GP, we only use ARD for the RBF kernel.
% Unlike for the flux power predictions, including ARD for the linear variance hyperparameter had no effect on the accuracy of the mean temperature predictions.
%
% The prediction and predicted uncertainty for a Gaussian process come from the mean and variance of the learned distribution, and can be written as the following closed-form expressions (for a full derivation, see \cite{2006gpml.book.....R}):
% \begin{equation}
%     \begin{split}
%         \mu(\boldsymbol{\theta}) &= \boldsymbol{\mathrm{K}}(\boldsymbol{\theta}, \mathcal{D})^\intercal \boldsymbol{\mathrm{K}}(\mathcal{D})^{-1} \boldsymbol{P_F}(\mathcal{D}); \\
%         \sigma^2(\boldsymbol{\theta}) &= k(\boldsymbol{\theta}, \boldsymbol{\theta}) - \boldsymbol{\mathrm{K}}(\boldsymbol{\theta}, \mathcal{D})^\intercal \boldsymbol{\mathrm{K}}(\mathcal{D})^{-1} \boldsymbol{\mathrm{K}}(\boldsymbol{\theta}, \mathcal{D}).
%     \end{split}
% \end{equation}
% where $\boldsymbol{\theta}$ are the new inputs for which a prediction is desired, $\mathcal{D}$ is the vector of training inputs, $\boldsymbol{P_F}(\mathcal{D})$ ($\boldsymbol{T}(\mathcal{D})$ for the mean temperature) is the vector of training outputs, $\boldsymbol{\mathrm{K}}(\mathcal{D})$ is the training data covariance matrix, and $\boldsymbol{\mathrm{K}}(\boldsymbol{\theta}, \mathcal{D})$ is the vector of covariances between the training data and new inputs.

% --------------------------------------------------------------------------------------------------
% --------------------------------------------------------------------------------------------------
\section{Inference Scheme and Likelihood Function}\label{sec:inference}

In this Section, we describe the inference scheme and likelihood function by which our cosmological parameter constraints are derived from our emulator, the eBOSS flux power spectrum \cite{2019JCAP...07..017C} and the mean IGM temperature.
The overall inference scheme is:
\begin{enumerate}
    \item Use the emulator to predict the flux power spectrum and IGM mean temperature for a set of input parameters (see Table~\ref{tab:emulatorparams}).
    \item Calculate a likelihood comparing these predictions to their observational counterparts from eBOSS \cite{2019JCAP...07..017C} and Ref.~\cite{2021MNRAS.506.4389G}.
    \item Use Cobaya \cite{2021JCAP...05..057T, 2019ascl.soft10019T} to run MCMC chains and compute posterior parameter constraints.
\end{enumerate}
Section~\ref{sec:fpsdata} discusses the flux power spectrum data, while Section~\ref{sec:t0data} discusses the IGM temperature data.
We derive our covariance matrix in Section~\ref{sec:theoryerror}.
Details of the likelihood calculation used in the MCMC sampling are given in Section~\ref{sec:likelihood}.
We validate our pipeline on simulated data in Section~\ref{sec:simdat}.

% ------------------------------------------------------------------------------------
% ------------------------------------------------------------------------------------

\subsection{Flux Power Spectrum Data}
\label{sec:fpsdata}
% percent error for four surveys (calculated from their available data):
%                z_min     z_max     k_min      k_max      overall
% chabanier:     6%(2.2) 18%(4.6)    6%(0.001)  14%(0.02)     7% (4-18% over z, 5-14% over k)
% day:          15%(2)   14%(4.2)   15%(0.003)   14%(0.1)     11%
% karacayli:     7%(2)   27%(4.6)   12%(0.005)   9%(0.1)      9% (5-27% over z, 7-13% over k)
% irsic:        12%(3)   15%(4.2)   10%(0.003)  16%(0.06)     10%

We use the observed \lya forest flux power spectrum from \cite{2019JCAP...07..017C}, which is based on the Baryon Oscillation Spectroscopic Survey (BOSS) and extended-BOSS (eBOSS) quasar samples \cite{2013AJ....145...10D, 2016AJ....151...44D}.
In \cite{2019JCAP...07..017C}, the BOSS/eBOSS quasar samples are refined to remove spectra that have not been visually inspected, and to remove spectra with broad absorption lines.
Sky lines and damped \lya absorbers (DLAs) are masked.
Our simulations include a realistic population of DLAs, which are masked in the same way.

The sample of \lya forests from the set of remaining quasar spectra is then further refined based on cuts to the spectral resolution, signal-to-noise ratio, number of masked pixels, and forest length, with a final sample of about $43,000$ spectra.
%Along with a windowing function to correct for the spectral resolution of the instrument, estimates for the noise and metal power (estimated using longer wavelength segments of the quasar spectra) are then subtracted to obtain the final \lya forest flux power spectrum.
The redshifts and scales covered by these observations set the redshift range and scales we use in our flux power spectrum emulator, namely $z=2.2-4.6$ (redshift bin size of $\Delta z = 0.2$), and $k\approx0.001-0.02$ s/km (over $35$ linearly spaced bins, $\Delta k = 5.42\times10^{-4}$ s/km).
Note, our emulator can easily be re-trained for the smaller scales probed by DESI.
The average uncertainty in the \lya forest flux power from Ref.~\cite{2019JCAP...07..017C} ranges from $\approx6\%$ at low redshifts or large scales, to $\approx16\%$ for high redshifts or small scales, and is often dominated by systematic uncertainty.

We apply correction terms to the \lya forest flux power spectrum predicted by our emulator to model DLAs and metal contamination.
We correct for DLAs using the template from Ref.~\cite{2018MNRAS.474.3032R}.
This allows us to account for differences in the DLA masking between our simulated pipeline and the observed pipeline.
An example would be DLAs, or Lyman limit systems (LLS), which are not detected in the observational pipeline due to low spectral signal-to-noise.
Note that our simulation includes a model that produces realistic populations of LLSs and DLAs, so the marginalised template allows for aspects in which the simulated model differs from the real Universe.
In \cite{2018MNRAS.474.3032R}, there are four parameters, with sub-DLAs separate from LLSs, and DLAs divided into two categories.
For each of the parameters, a redshift and scale dependent correction is applied, where a positive (negative) value for the parameter implies that our simulation has underestimated (overestimated) the number of absorbers in that category. 
We found that in practice our dataset was unable to measure separately all four of the column density bins.
We thus simplify our likelihood by using only two additional free parameters, one parameter covering sub-DLAs and LLS, $\alpha_{\mathrm{lls}}$, and one parameter covering DLAs, $\alpha_{\mathrm{dla}}$.
$\alpha_{\mathrm{lls}}$ covers column densities between $1.6\times10^{17} - 10^{20}$ cm$^{-2}$), and $\alpha_{\mathrm{dla}}$ covers $10^{21}-10^{22.5}$  cm$^{-2}$. 

We account for correlated Si~{\sc{iii}} absorption within the \lya~forest following Ref.~\cite{2006ApJS..163...80M}.
Our likelihood includes an additional nuisance parameter, $f_\mathrm{SiIII}$, which measures the amplitude of the metal contamination. 

% -------------------------------------------------------------------------------------------------

\subsection{IGM Mean Temperature Data}\label{sec:t0data}

We use the mean IGM temperatures from Ref.~\cite{2021MNRAS.506.4389G}, derived from simulation modeling of high resolution quasar spectra from the KODIAQ survey \cite{2017AJ....154..114O}.
Ultimately the dataset is a relatively small, visually inspected set of high resolution quasar spectra.
Importantly, these spectra are independent of the eBOSS quasar sample, justifying our choice of separate likelihood functions.
We include IGM mean temperature data for $z=2.2-3.8$, for consistency with the available \lya forest flux power data.
The average uncertainty for this data set is $\approx10\%$, whereas our mean temperature emulator has an average uncertainty of $\sim 1\%$.
Ref.~\cite{2021MNRAS.506.4389G} provides mean temperatures derived from four different statistics: the \lya forest flux power spectrum, curvature, wavelet decomposition, and Doppler width distribution.
We use the \lya forest flux power spectrum derived temperatures in the main body of this work, but show results using these other data sets in Appendix~\ref{sec:t0-only}.

%In \cite{2021MNRAS.506.4389G}, their co-added quasar spectra are manually checked, and the sample is refined to remove spectra that do not contain \lya forest, do contain DLAs or sub-DLAs, have large gaps in the spectra, or do not meet a signal-to-noise ratio cut.
% A cut is also made to remove regions of the spectra that may lie near to a quasar proximity zone.
%Metal lines are removed by fitting Voigt profiles to the spectra and removing features with Doppler widths $b \leq 8$ km s$^{-1}$, which are assumed to be metal lines.
To derive temperatures from the observed quasar spectra, Ref.~\cite{2021MNRAS.506.4389G} calculated several summary statistics and compared them to those derived from spectra drawn from simulations.
The simulations they used were similar in resolution to our HF suite (gas mass resolution of $\sim10^5$ M$_{\odot}$), though much smaller in volume ($10$ Mpc/h box side length).
Because these observed mean temperatures are themselves derived using a suite of simulations, it would be optimal to remove the middle step in future work, i.e. calculate mean temperatures using the observed small-scale 1D flux power spectrum and our own set of simulations.

% ---------------------------------------------------------------------------

\subsection{Covariance Matrix}
\label{sec:theoryerror}

\begin{figure}
    \centering
    \includegraphics[width=\textwidth]{figures/errors_loo.pdf}
    \caption{\label{fig:covariance_loo}
    Diagonal elements of the covariance matrix as a function of scale and for selected redshift bins.
    Shown are the square root of the diagonal elements of the eBOSS covariance matrix and the diagonal cosmic variance error estimated from leave-one-out errors, $\boldsymbol{\sigma}_{CV}$.}
\end{figure}

In this Section, we derive the covariance matrix, $\boldsymbol{K}$, that is used for our inference.
We decompose $\boldsymbol{K}$ as:
\begin{equation}
    \boldsymbol{K} = \boldsymbol{K}_\mathrm{BOSS} + \boldsymbol{\sigma}_{GP}(\boldsymbol{p}) \cdot \boldsymbol{\sigma}_{GP}^T (\boldsymbol{p}) + \boldsymbol{\sigma}_{CV} \cdot \boldsymbol{\sigma}_{CV}^T \,.
    \label{eq:covariance}
\end{equation}
Here, $\boldsymbol{K}_\mathrm{BOSS}$ is the covariance matrix from the eBOSS pipeline \cite{2019JCAP...07..017C}, and is the largest term in the covariance matrix on most scales.
We also add two extra terms which model theoretical error in our model.
$\boldsymbol{\sigma}_{GP}(\boldsymbol{p})$ is the parameter dependent estimate of the interpolation error from the Gaussian process.
We found that in some cases when a parameter was poorly constrained this could unphysically drive the chain towards the edge of parameter space where the interpolation error was large.
We thus choose to omit it from the overall covariance matrix.
Appendix~\ref{sec:loovsgperr} shows that its addition has a small effect on our final results.

The second theoretical error in our simulation suite (which dominates) is $\boldsymbol{\sigma}_{CV}$, which models residual sample variance from the finite box size, analogous to cosmic variance from the finite cosmological horizon\footnote{Ref.~\cite{2023ApJ...944..223P} reduced sample variance by interpolating the parameters of a higher order polynomial, rather than fitting the binned flux power spectrum directly.
Our emulator is much less affected by sample variance as our simulated volume is $8$ times larger.}.
We include an estimate of sample variance using the leave-one-out errors discussed in Ref.~\cite{2023simsuite}, a technique made possible by the inclusion of $h$ in our simulation suite.
The Hubble parameter does not directly affect the gravitational evolution in our simulations due to Gadget's use of Mpc/h units, and Ref.~\cite{2023simsuite} showed that the effect on the thermal history is small on the scales probed by eBOSS.
However, in our parameterization, $h$ also changes $\Omega_M$ \footnote{$\Omega_M h^2$ is a separate parameter and so kept fixed when varying $h$.} \maf{should this be $\Omega_M h^2$?} \spb{No, the point is that varying h for fixed OmegaMh2 also changes OmegaM} and so the conversion of wavenumbers from h/Mpc to s/km.
Individual Fourier modes thus move between bins depending on the value of $h$, mimicking the sample variance from different initial phases.
We thus approximate $\boldsymbol{\sigma}_{CV}$ with the averaged variance of the leave-one-out errors using the low fidelity simulations.
Leave-one-out errors are found by building a reduced emulator, which is trained on all but a single sample, then evaluating the prediction accuracy for that left-out sample using the reduced emulator.
This is then repeated, such that every sample takes a turn being left out (see Figure~2 of \cite{2023simsuite}):
\begin{equation}
    \boldsymbol{\sigma}^2_{CV}  = \frac{1}{N_{LF}}\Sigma_i \left(P_F^\mathrm{Predict}(k, z, p_i) - P_F^\mathrm{True}(k, z, p_i)\right)^2\,.
\end{equation}
Here the sum runs over all simulated low-fidelity parameter sets $p_i$ and $N_{LF}$ is the number of low-fidelity simulations. 

Figure~\ref{fig:covariance_loo} shows the magnitude of the $\boldsymbol{\sigma}_{CV}$ term compared to the eBOSS errors.
$\boldsymbol{\sigma}_{CV}$ is significant only on the largest scales, $k < 2.5 \times 10^{-3}$ s/km, as expected from an effect due to finite box size.
In addition, there is a significant redshift dependence: $\boldsymbol{\sigma}_{CV}$ is large compared to the eBOSS errors only for $2.8 < z < 3.4$.
For clarity, Figure~\ref{fig:covariance_loo} shows only $z=2.8$ and $z=3.4$. However, we have verified that $\boldsymbol{\sigma}_{CV}$ reduces for lower redshifts. For the redshift range $4.2 > z>3.4$, $\boldsymbol{\sigma}_{CV}$ remains approximately constant, but becomes less significant as the eBOSS statistical errors increase. \maf{not sure I totally follow this sentense in the context of the figure} \spb{Is that better?}.
These details reveal the physical source of this large-scale variance.
The relevant scale is close to the $20$ Mpc/h size of the Helium reionization bubbles, and the relevant redshift range is when our model performs helium reionization.
Helium reionization bubbles are placed randomly around rare large halos, which creates sample variance in a finite box.

\subsection{Likelihood}\label{sec:likelihood}

We use a log normal likelihood summed over all redshifts and, for the flux power, all scale bins:
\begin{equation}
    \mathrm{log}\mathcal{L} = -\frac{1}{2} \sum_{z=2.2}^{z=4.6} \left(\boldsymbol{P}_F^{\mathrm{diff}} \cdot \boldsymbol{K}^{-1} \cdot \boldsymbol{P}_F^{\mathrm{diff}} + \mathrm{det}(\boldsymbol{K})\right)
    \label{eq:likelihood}
\end{equation}
where $\boldsymbol{P}_F^{\mathrm{diff}} = \boldsymbol{P}_F^{\mathrm{sim}} - \boldsymbol{P}_F^{\mathrm{obs}}$ is the vector difference between the simulation prediction and the observation.
The covariance matrix, $\boldsymbol{K}$, is described in Equation~\ref{eq:covariance}. 

The likelihood for the IGM mean temperature is similar, but single valued per redshift.
We compute the \lya forest flux power and IGM mean temperature likelihoods separately.
Because the flux power spectrum has larger magnitude likelihoods (due to the extra dimension of information, $k$, and additional redshift bins), the mean temperature likelihood is scaled by a factor of $\sim 10$, to ensure both information sources contribute roughly equally to the posterior.

We make use of the Cobaya package \cite{2021JCAP...05..057T, 2019ascl.soft10019T, 2013PhRvD..87j3529L, 2002PhRvD..66j3511L} to run MCMC chains using this likelihood.
The MCMC sampler uses the Metropolis method discussed in \cite{2013PhRvD..87j3529L}, and uses a Gaussian + exponential proposal distribution that dynamically learns the proposal covariance.
Convergence is determined using the Gelman-Rubin statistic, $R$, also detailed in \cite{2013PhRvD..87j3529L}.
The chains presented here were all run until a convergence of $R-1 < 0.01$, with results plotted for those chains for samples at $R-1 < 1$.

\subsubsection{Priors}

We use the parameter limits shown in Table~\ref{tab:emulatorparams}.
As we showed in Ref.~\cite{2023simsuite}, the AGN feedback parameter $\epsilon_{AGN}$ has minimal effect on the \Lya forest 1D flux power spectrum.
Preliminary chains indicated that it is indeed poorly constrained by the data and has minimal correlations with other parameters.
We use a strong Gaussian prior with $\mu = 0.05$ and $\sigma = 0.005$, which dominates over data constraints, and will omit constraints on $\epsilon_{AGN}$ from our results.
We also place a weak Gaussian prior on the Hubble parameter, $h$, with $\mu = 0.70$ and $\sigma = 0.015$, as it is weakly constrained and this prior avoids the inference straying into areas near the edge of parameter volume where the emulation is less accurate.
For all other parameters we use uniform priors within the parameter limits.

\subsection{Inference Using Simulation Data}\label{sec:simdat}

In this section we test our inference framework with simulation outputs in place of the observational data, confirming that we recover the known input parameters.
We first used the flux power spectrum from one of the three high fidelity simulations, and confirmed that the maximum likelihood was indeed at the input parameter values for all parameters.
All input parameters were recovered to better than one sigma.

\begin{figure}
    \centering
    \includegraphics[width=\textwidth]{figures/simdat-seed.pdf}
    \caption{\label{fig:simdat_posteriors}
    \maf{if not too much trouble, maybe change the dashed line color -- blue might work} \spb{How's that? I went with black so that there weren't too many colours floating around.}
    Posteriors using mock data, a simulation output with a different initial seed to the main PRIYA suite.
    The true parameter values for the input are indicated by the red dashed lines.
    Three chains are used: `Seed FPS' uses the default error model, with the eBOSS covariance and leave-one-out errors.
    `Seed FPS + GPERR' adds the covariance from the Gaussian Process.
    `Seed FPS + $T_0$' uses the default error model but supplements the flux power spectrum data with information from the mean temperature.
    $v_{scale}$ is the Hubble parameter $h$, re-labelled for reasons explained in the text.}
\end{figure}

We next ran chains using data from a low-fidelity simulation with a different random seed from the main PRIYA suite.
For these runs only, we used an emulator built using the low fidelity suite.
This test was designed to quantify whether the finite box size of our simulations can affect our parameter constraints.
Figure~\ref{fig:simdat_posteriors} shows the results, with dashed red lines indicating the correct parameters.

We have performed three runs.
The first (Seed LOO) is our preferred error model, using the eBOSS covariance and a leave-one-out (LOO) error term.
The second adds an error term for the expected interpolation error from the Gaussian Process (Seed LOO + GP).
Both of these runs use only information from the eBOSS flux power spectrum and thus do not provide strong constraints on the parameters of helium reionization.
We therefore run another chain including constraints from the mean temperature.

In all three runs, the optical depth parameters, $\tau_0$ and $d\tau_0$, are tightly constrained around the true value in our pipeline, despite the effect of a different structure seed. The GPERR chain increases the uncertainty, especially on $d\tau_0$, but does not bias the measurement.
The best estimate comes from the flux power spectrum data alone (Seed FPS).
We also consistently recover the true values of the cosmological parameters $n_P$ and $A_P$, although $A_P$ is about $1-\sigma$ high for the chain which includes the mean temperature.
This is not unreasonable as we deliberately constructed our test data, with a different structure seed, to be different from the training data.
$\Omega_M h^2$ is poorly constrained in all chains, as expected given that our prior volume includes only a narrow range for $\Omega_M h^2$, motivated by Planck results.
All parts of the prior range are within the $1-\sigma$ posteriors.

The redshift of hydrogen reionization, $z_{HI}$, is estimated from the mean IGM temperature at $z > 3.6$ or from a large-scale increase in the flux power spectrum at $z > 4$ (see Ref.~\cite{2023simsuite}).
The second effect is due to a scale-dependent bias arising from placement of the reionization bubbles \cite{Montero:2019}.
Figure~\ref{fig:simdat_posteriors} indicates that this bias is sensitive to sample variance from the finite box, and so the hydrogen reionization redshift is not well measured by the flux power spectrum data alone.
The three parameters which govern Helium reionization, $z_i^{HeII}$, $z_f^{HeII}$ and $\alpha_q$, are well constrained by the mean temperature data.
The runs which do not include mean temperature data have a preference for a larger $\alpha_q$ than the input value.
As discussed above, the main effect of a different structure seed is through the placement of Helium reionization bubbles.
$\alpha_q$ is thus measured using a similar scale-dependent bias as $z_{HI}$, and so is slightly sensitive to the finite box size in the same way.
However, the mean IGM temperature is sensitive to $\alpha_q$ through the peak temperature during Helium reionization, and thus the chains including it correctly infer $\alpha_q$.

The chain including Gaussian Process errors sometimes produced incorrect parameter inferences, notably in $\alpha_q$.
It seems that GP errors create an implicit prior which sometimes shifts the results towards regions of higher interpolation accuracy.
The simulated data seems to exaggerate this effect, perhaps because of the exact choice of simulated parameters near the edge of the prior volume.
We show in Appendix~\ref{sec:loovsgperr} that the posteriors from eBOSS data are not significantly changed by including GP error.
Nevertheless, to be conservative our main results are reported with chains run omitting GP errors from the likelihood.

As discussed above, the Hubble parameter, $h$, does not affect the evolution of our simulations except through its effect on $\Omega_M$ (at fixed $\Omega_M h^2$) and thus the scaling between velocity (km/s) and comoving (Mpc/h) units.
Constraints on $h$ ($v_\mathrm{scale}$) are incorrect in the chains shown in Figure~\ref{fig:simdat_posteriors}, driven by sample variance in the finite box.
We confirmed that gradients of the likelihood with respect to $h$ are largest for the largest scales, particularly the first $4$ $k$-bins.
We computed the $\chi^2$ per degree of freedom, which was $\sim 0.9$ for $h = 0.65$ and $\chi^2 \sim 1$ for the true input value, indicating over-fitting to the noise created by different structure seeds.
We confirmed that fixing $h$ to the known true value results in very small changes to the other parameters and their confidence intervals.
There is thus zero cosmological information in our $h$ constraints.
In order to avoid unwarranted conclusions, we will henceforth relabel $h$ as $v_\mathrm{scale}$, emphasising that it merely controls the mapping between the native Fourier-space binning of the simulations and the observed velocity space of the spectra, and its inference is dominated by modelling error from sample variance.
\section{Results}\label{sec:results}

In this Section, we report posterior constraints on the parameters listed in Table~\ref{tab:emulatorparams}.
Section~\ref{sec:cosmo} discusses the results for those parameters which are most strongly constrained by the flux power spectrum data.
These are: the optical depth parameters, $\tau_0$ and $d\tau_0$, the power spectrum parameters, $n_P$ and $A_p$, and the \textbf{matter density} $\Omega_M h^2$ \textbf{(which is also the growth function during matter domination)}.
Section~\ref{sec:astro} then discusses the constraints on the other parameters, and shows the best fit to the mean IGM temperature data.
These are the three parameters defining the He~{\sc{ii}} reionization model, z$^{\text{He~{\sc ii}}}_i$, z$^{\text{He~{\sc ii}}}_f$, and $\alpha_q$; the parameter for the midpoint of H~{\sc{i}} reionization, z$^{\text{H~{\sc i}}}$, the strong absorber models, $\alpha_{LLS}$ and $\alpha_{DLA}$), the Silicon III correction (fSiIII) and the velocity to distance scale parameter $v_{scale}$.
The same chains are used in all sections: we split parameters into two sections merely for readability.
We show the full corner plot, containing all constrained parameters, in Appendix~\ref{sec:full_posteriors}.
Table~\ref{table:parameters} shows posterior parameter constraints, including the derived parameters $A_s$ and $\sigma_8$. 

% --------------------------------------------------------------------------------------------------

\subsection{Cosmological Parameters}\label{sec:cosmo}

\begin{figure}
    \centering
    \includegraphics[width=0.75\textwidth]{figures/cosmo_corner.pdf}
    \caption{\label{fig:cosmo_corner}
    Posteriors for the optical depth and power spectrum parameters, $\tau_0$, $d\tau_0$, $n_P$, $A_P$, and $\Omega_M h^2$.
    Results are from three MCMC chains.
    \textbf{`FPS $z=2.2-4.6$' (gold) uses the full redshift range eBOSS flux power spectrum dataset, `FPS $z=2.6-4.6$' (blue) uses a reduced redshift range eBOSS dataset flux power spectrum dataset, which removes the internal tension (Section~\protect\ref{sec:tension}).
    The third chain, `FPS $+ T_0, z=2.6-4.6$' (red), uses the limited range eBOSS dataset but adds the mean IGM temperature constraints.}
    Our preferred cosmological constraints are from `FPS $z=2.6-4.6$'.
    }
\end{figure}

%\begin{table}
%\centering
%\def\arraystretch{1.2}
%\begin{tabular} {| l | c | c |}
%\hline
% Parameter &  68\% limits &  95\% limits\\
%\hline
%$d\tau_0        $ & $-0.240\pm 0.030           $    & $-0.240^{+0.058}_{-0.059}  $   \\
%$\tau_0         $ & $1.214^{+0.013}_{-0.010}   $    & $1.214^{+0.022}_{-0.024}   $   \\
%$n_\mathrm{P}   $ & $0.891\pm 0.011            $    & $0.891^{+0.021}_{-0.020}   $   \\
%$A_\mathrm{P}/10^{-9}$ & $< 1.28 $    & $< 1.36$   \\
%$z^{HeII}_i     $ & $> 4.07                    $    & $> 4.02                    $   \\
%$z^{HeII}_f     $ & $< 2.68                    $    & $< 2.80                    $   \\
%$\alpha_{q}     $ & $> 2.41                    $    & $> 2.28                    $   \\
%$v_\mathrm{scale}$ & $0.6943^{+0.0070}_{-0.0086}$   & $0.694^{+0.018}_{-0.016}  $   \\
%$\Omega_M h^2   $ & $0.14143^{+0.00063}_{-0.00097}$ & $< 0.143                   $   \\
%$z^{HI}         $ & $> 7.79                    $    & $> 7.54                    $   \\
%$\alpha_{lls}   $ & $0.039^{+0.017}_{-0.019}   $    & $0.039^{+0.037}_{-0.034}   $   \\
%$\alpha_{dla}   $ & $-0.0073\pm 0.0032         $    & $-0.0073^{+0.0062}_{-0.0064}$  \\
%$f_\mathrm{SiIII}         $ & $0.00881\pm 0.00040        $    & $0.00881^{+0.00076}_{-0.00077}$\\
%\hline
%$A_\mathrm{s}/10^{-9}      $ & $1.709^{+0.063}_{-0.089}$ & $1.71^{+0.16}_{-0.15}$\\
%$\sigma_8                  $ & $0.713^{+0.012}_{-0.018}$  & $0.713^{+0.032}_{-0.028}$\\
%\hline
%\end{tabular}
%\caption{\label{table:z22z46parameters}
%Constraints from the flux power spectrum using the full redshift range, $z = 2.2 - 4.6$. Maximum posterior values, $68\%$ and $95\%$ confidence limits are shown. $A_s$ and $\sigma_8$ are derived.
%}
%\end{table}

%\begin{table}
%	\centering
%      \def\arraystretch{1.2}
%\begin{tabular} {| l | c | c | c | c|}
%\hline
% Parameter &  FPS  (68\%) &  FPS  (95\%) &  FPS + $T_0$ (68\%) &  FPS + $T_0$  (95\%) \\
%\hline
%$d\tau_0        $ & $-0.006\pm 0.037           $ & $-0.006^{+0.071}_{-0.076}  $  & $0.028\pm 0.036            $ & %$0.028^{+0.070}_{-0.071}   $\\
%$\tau_0         $ & $1.098^{+0.016}_{-0.020}   $ & $1.098^{+0.036}_{-0.033}   $  & $1.093\pm 0.017            $ & %$1.093^{+0.034}_{-0.033}   $\\
%$n_\mathrm{P}   $ & $> 0.969                   $ & $> 0.949                   $  & $0.972^{+0.018}_{-0.0095}  $ & $> 0.948                   $\\
%$A_\mathrm{P}/10^{-9}$ & $1.56\pm 0.12         $ & $1.56^{+0.25}_{-0.23}$        & $1.385^{+0.083}_{-0.11}$   & $< 1.56       $               \\
%$z^{HeII}_i     $ & $> 4.04                    $ & $> 3.95                    $  & $4.037^{+0.052}_{-0.024}   $ & $> 3.97                    $\\
%$z^{HeII}_f     $ & $< 2.76                    $ & $< 2.98                    $  & $2.718\pm 0.044            $ & $2.718^{+0.087}_{-0.087}   $\\
%$\alpha_{q}     $ & $> 2.32                    $ & $> 2.03                    $  & $1.404^{+0.038}_{-0.092}   $ & $< 1.52                    $\\
%$v_\mathrm{scale}$ & $0.691^{+0.011}_{-0.0091}  $ & $0.691^{+0.021}_{-0.022}$    & $0.684^{+0.016}_{-0.0096} $  & $0.684^{+0.021}_{-0.026}   $\\
%$\Omega_M h^2   $ & $0.1425^{+0.0011}_{-0.0015}$ & $0.1425^{+0.0024}_{-0.0024}$ & $0.1427^{+0.0013}_{-0.0018}$ & ---                         \\
%$z^{HI}         $ & $> 7.61                    $ & $> 7.24                    $ & $7.58^{+0.37}_{-0.16}      $ & $> 7.09                    $\\
%$\alpha_{lls}   $ & $0.162\pm 0.029            $ & $0.162^{+0.056}_{-0.057}   $ & $0.191\pm 0.028            $ & $0.191^{+0.054}_{-0.056}   $\\
%$\alpha_{dla}   $ & $-0.0038\pm 0.0060         $ & $-0.0038^{+0.012}_{-0.012}  $ & $-0.0135\pm 0.0054         $ & $-0.014^{+0.010}_{-0.011}  $\\
%$fSiIII         $ & $0.00966\pm 0.00051        $ & $0.00966^{+0.0010}_{-0.0010}$ & $0.00962\pm 0.00051        $ & $0.00962^{+0.0010}_{-0.00099} $ \\
%\hline
%$A_\mathrm{s}/10^{-9}      $ & $1.67^{+0.12}_{-0.13}$ & $1.67^{+0.27}_{-0.24}$ & $1.495^{+0.094}_{-0.12}$ & $1.495^{+0.21}_{-0.20}$\\
%$\sigma_8$ & $0.733\pm 0.026            $ & $0.733^{+0.052}_{-0.049}$ &  $0.696^{+0.021}_{-0.025}   $ & $0.696^{+0.045}_{-0.043}$ \\
%\hline
%\end{tabular}
%\caption{\label{table:parameters}
%Constraints from the flux power spectrum (FPS) and flux power spectrum and mean IGM temperature (FPS + $T_0$) using the reduced redshift range, $z = 2.6 - 4.6$. Maximum posterior values, $68\%$ and $95\%$ confidence limits are shown. $A_s$ and $\sigma_8$ are derived.
%    }
%\end{table}

% \begin{table}
% 	\centering
%       \def\arraystretch{1.6}
% \begin{tabular} {| l | c | c | c|}
% \hline
% Parameter &  FPS $z>2.6$ 68\% (95\%) &  FPS + $T_0$ 68\%, (95\%) & FPS $z>2.2$ 68\%, (95\%) \\
% \hline
% $d\tau_0        $ & $-0.006\pm 0.037$  $\left(^{+0.071}_{-0.076}\right)$                   & $0.028\pm 0.036$            $\left(^{+0.070}_{-0.071}\right)$                   & $-0.240\pm 0.030$     $\left(^{+0.058}_{-0.059}  \right)$   \\
% $\tau_0         $ & $1.098^{+0.016}_{-0.020}$  $\left( ^{+0.036}_{-0.033}\right)$            & $1.093\pm 0.017$       $\left(^{+0.034}_{-0.033}\right)$                   & $1.214^{+0.013}_{-0.010}$   $\left(^{+0.022}_{-0.024}\right)$   \\
% $n_\mathrm{P}   $ & $> 0.969                   $  ($> 0.949                   $)  & $0.972^{+0.018}_{-0.0095}$  ($> 0.948   $)    & $0.891\pm 0.011$            $\left(^{+0.021}_{-0.020}\right)$   \\
% $A_\mathrm{P}/10^{-9}$ & $1.56\pm 0.12$        $\left(^{+0.25}_{-0.23}\right)$        & $1.385^{+0.083}_{-0.11}$   ($< 1.56 $)                   &           $< 1.28 $  ($< 1.36$)   \\
% $\Omega_M h^2   $ & $0.1425^{+0.0011}_{-0.0015}$ $(^{+0.0024}_{-0.0024})$       & $0.1427^{+0.0013}_{-0.0018}$  (---)                              & $0.14143^{+0.00063}_{-0.00097}$ ($< 0.143$)   \\
% $z^{HeII}_i     $ & $> 4.04                    $ $> 3.95                    $  & $4.037^{+0.052}_{-0.024}   $ $(> 3.97 ) $    & $> 4.07                    $  $(> 4.02)$   \\
% $z^{HeII}_f     $ & $< 2.76                    $  $(< 2.98)$                   & $2.718\pm 0.044            $  $\left(^{+0.087}_{-0.087}\right)$    & $< 2.68                    $    $(< 2.80 )$   \\
% $\alpha_{q}     $ & $> 2.32                    $  $(> 2.03)$                   & $1.404^{+0.038}_{-0.092}   $  $(< 1.52 )$              & $> 2.41                    $    $(> 2.28 )                   $   \\
% $z^{HI}         $ & $> 7.61                    $ ($> 7.24 $) & $7.58^{+0.37}_{-0.16}      $ ($> 7.09 $)     & $> 7.79                    $    $(> 7.54 )$   \\
% $v_\mathrm{scale}$ & $0.691^{+0.011}_{-0.0091}  $  $\left(^{+0.021}_{-0.022}\right)$    & $0.684^{+0.016}_{-0.0096} $  $\left(^{+0.021}_{-0.026}\right)$    & $0.6943^{+0.0070}_{-0.0086}$  $\left(^{+0.018}_{-0.016} \right)$     \\
% $\alpha_{lls}   $ & $0.162\pm 0.029   $ $\left(^{+0.056}_{-0.057}\right)$          & $0.191\pm 0.028            $  $\left(^{+0.054}_{-0.056}\right)   $     & $0.039^{+0.017}_{-0.019}   $  $\left(^{+0.037}_{-0.034}\right)$   \\
% $\alpha_{dla}   $ & $-0.0038\pm 0.0060         $ $\left(^{+0.012}_{-0.012}\right)$ & $-0.0135\pm 0.0054         $ $\left(^{+0.010}_{-0.011}\right)  $    & $-0.0073\pm 0.0032         $   $\left(^{+0.0062}_{-0.0064}\right)$  \\
% $f_{SiIII}/10^{-3} $ & $9.66\pm 0.51              $ $\left(^{+1.0}_{-1.0}\right)$              & $9.62\pm 0.51        $       $\left(^{+1.0}_{-0.99}\right) $          & $8.81\pm 0.40        $  $\left(^{+0.76}_{-0.77}\right)$\\
% \hline
% $A_\mathrm{s}/10^{-9}      $ & $1.67^{+0.12}_{-0.13}$ $\left(^{+0.27}_{-0.24}\right)$ & $1.495^{+0.094}_{-0.12}$ $\left(^{+0.21}_{-0.20}\right)$  & $1.709^{+0.063}_{-0.089}$ $\left(^{+0.16}_{-0.15}\right)$   \\
% $\sigma_8$ & $0.733\pm 0.026            $ $\left(^{+0.052}_{-0.049}\right)$ &  $0.696^{+0.021}_{-0.025}   $ $\left(^{+0.045}_{-0.043}\right)$ & $0.713^{+0.012}_{-0.018}$  $\left(^{+0.032}_{-0.028}\right)$  \\
% \hline
% \end{tabular}
% \caption{\label{table:parameters}
% Posterior parameter constraints, including the derived parameters $A_s$ and $\sigma_8$. Maximum posterior values, and $68\%$ confidence limits are shown, with 95\% confidence intervals in brackets. Each column shows a separate chain, from left to right: fits to the flux power spectrum alone from the reduced redshift range $z=2.6 - 4.6$, fits to the flux power spectrum from the reduced redshift range $z=2.6 - 4.6$ and to the mean IGM temperature, and fits to the flux power spectrum alone from the full redshift range $z=2.2 - 4.6$. Single sided limits are shown when one bound is larger than the prior volume of the emulator.
% }
% \end{table}

\begin{table}
	\centering
      \def\arraystretch{1.6}
\begin{tabular} {| l | c | c | c|}
\hline
 & FPS $z>2.6$ & FPS + $T_0$ & FPS $z>2.2$ \\
Parameter & 68\% (95\%) & 68\%, (95\%) & 68\%, (95\%)
\\
\hline
$d\tau_0        $ & $-0.006\pm 0.037$  $\left(^{+0.071}_{-0.076}\right)$                   & $0.017\pm 0.039$            $\left(^{+0.077}_{-0.076}\right)$                   & $-0.240\pm 0.030$     $\left(^{+0.058}_{-0.059}  \right)$   \\
$\tau_0         $ & $1.098^{+0.016}_{-0.020}$  $\left( ^{+0.036}_{-0.033}\right)$            & $1.095\pm 0.018$       $\left(^{+0.035}_{-0.033}\right)$                   & $1.214^{+0.013}_{-0.010}$   $\left(^{+0.022}_{-0.024}\right)$   \\
$n_\mathrm{P}   $ & $> 0.969                   $  ($> 0.949                   $)  & $0.973^{+0.019}_{-0.0092}$  ($> 0.948   $)    & $0.891\pm 0.011$            $\left(^{+0.021}_{-0.020}\right)$   \\
$A_\mathrm{P}/10^{-9}$ & $1.56\pm 0.12$        $\left(^{+0.25}_{-0.23}\right)$        & $1.423^{+0.097}_{-0.13}$   ($^{+0.20}_{-0.21}$)                   &           $< 1.28 $  ($< 1.36$)   \\
$\Omega_M h^2   $ & $0.1425^{+0.0011}_{-0.0015}$ $(^{+0.0024}_{-0.0024})$       & $0.1426^{+0.0012}_{-0.0017}$  ($^{+0.0025}_{-0.0026}$)                  & $0.1414^{+0.0006}_{-0.0010}$ ($< 0.143$)   \\
$z^{HeII}_i     $ & $> 4.04                    $ $(> 3.95)                    $  & $> 4.04$   $(> 3.97 ) $    & $> 4.07                    $  $(> 4.02)$   \\
$z^{HeII}_f     $ & $< 2.76                    $  $(< 2.98)$                   & $< 2.77$  $(< 2.90)$           & $< 2.68                    $    $(< 2.80 )$   \\
$\alpha_{q}     $ & $> 2.32                    $  $(> 2.03)$                   & $1.60^{+0.13}_{-0.20}$  $(< 1.90 )$              & $> 2.41                    $    $(> 2.28 )                   $   \\
$z^{HI}         $ & $> 7.61                    $ ($> 7.24 $)                    & $> 7.53$ ($> 7.16 $)                            & $> 7.79                    $    $(> 7.54 )$   \\
$v_\mathrm{scale}$ & $0.691^{+0.011}_{-0.009}  $  $\left(^{+0.021}_{-0.022}\right)$    & $0.686^{+0.014}_{-0.0089}$  $\left(^{+0.020}_{-0.026}\right)$    & $0.694^{+0.007}_{-0.009}$  $\left(^{+0.018}_{-0.016} \right)$     \\
$\alpha_{lls}   $ & $0.162\pm 0.029   $ $\left(^{+0.056}_{-0.057}\right)$          & $0.185\pm 0.029$  $\left(^{+0.055}_{-0.058}\right)   $     & $0.039^{+0.017}_{-0.019}   $  $\left(^{+0.037}_{-0.034}\right)$   \\
$\alpha_{dla}/10^{-2}   $ & $-0.4\pm 0.6         $ $\left(^{+1.2}_{-1.2}\right)$ & $-1.14\pm 0.58$ $\left(\pm 1.1\right)  $    & $-0.7\pm 0.3         $   $\left(^{+0.6}_{-0.6}\right)$  \\
$f_{SiIII}/10^{-3} $ & $9.66\pm 0.51              $ $\left(^{+1.0}_{-1.0}\right)$              & $9.64\pm 0.52        $       $\left(\pm 1\right) $          & $8.81\pm 0.40        $  $\left(^{+0.76}_{-0.77}\right)$\\
\hline
$A_\mathrm{s}/10^{-9}      $ & $1.67^{+0.12}_{-0.13}$ $\left(^{+0.27}_{-0.24}\right)$ & $1.54^{+0.10}_{-0.13}$ $\left(^{+0.23}_{-0.22}\right)$  & $1.71^{+0.06}_{-0.09}$ $\left(^{+0.16}_{-0.15}\right)$   \\
$\sigma_8$ & $0.733\pm 0.026            $ $\left(^{+0.052}_{-0.049}\right)$ &  $0.704^{+0.023}_{-0.028}   $ $\left(^{+0.049}_{-0.047}\right)$ & $0.713^{+0.012}_{-0.018}$  $\left(^{+0.032}_{-0.028}\right)$  \\
\hline
\end{tabular}
\caption{\label{table:parameters}
Posterior parameter constraints, including the derived parameters $A_s$ and $\sigma_8$.
Maximum posterior values, and $68\%$ confidence limits are shown, with $95\%$ confidence intervals in brackets.
Each column shows a separate chain, from left to right: fits to the flux power spectrum alone from the reduced redshift range $z=2.6 - 4.6$, fits to the flux power spectrum from the reduced redshift range $z=2.6 - 4.6$ and the mean IGM temperature, and fits to the flux power spectrum alone from the full redshift range $z=2.2 - 4.6$.
Single sided limits are shown when one bound is larger than the prior volume of the emulator.
}
\end{table}


\begin{figure}
    \centering
    \includegraphics[width=\textwidth]{figures/fps_data_fit.pdf}
    \caption{\label{fig:fps_data}
    Observed \lya forest flux power spectrum \cite{2019JCAP...07..017C}, from $z=4.6$ to $z=2.2$ (black lines and circles, with shading corresponding to one sigma uncertainty).
    Also shown are three predictions for the \lya forest flux power spectrum from our multi-fidelity emulator corresponding to the maximum posterior input parameters compiled in Table~\ref{table:parameters}.
    The negative of the log-likelihood for these fits is compiled in Table~\ref{table:chi2}.
    }
\end{figure}

Figure~\ref{fig:cosmo_corner} shows the results of our chains for the cosmological parameters.
We show three MCMC chains.
Two chains are fit to the eBOSS flux power spectrum data only.
The first fits to the full redshift range measured by eBOSS, $z=2.2 - 4.6$, while the second fits a limited redshift range $z=2.6 - 4.6$.
The third chain uses the limited redshift range eBOSS dataset but adds the mean IGM temperature likelihood.
The chain including the $z < 2.6$ data prefers lower $n_P$, lower $A_P$ and higher $\tau_0$ than the reduced redshift range.
Figure~\ref{fig:fps_data} shows that the shift in posterior parameters is driven by the fit.
The best-fit flux power spectrum to the data at $z \geq 2.6$ is a poor fit to the flux power spectra measured at $z=2.2$ and $z=2.4$.
Since the lowest redshift bins have the smallest statistical error, when they are included they drive the best-fit flux power spectrum to a region which is a poorer fit to the higher redshift data.
We confirmed that a chain which included the $z=2.4$ bin but not the $z=2.2$ bin produced posterior constraints mid-way between the chain including $z=2.2-4.6$ and the chain including $z=2.6-4.6$.
Table~\ref{table:chi2} shows this quantitatively.
The chains which fit to $z > 2.6$ are a poor fit to the lowest two redshift bins.
The chain fitting to $z=2.2$ is a better fit to these bins, at the cost of an overall worse fit to most higher redshifts.
The total $\chi^2$ per degree of freedom for the reduced redshift chains is close to $1.03$, indicating a good fit, while the full reduced range has a $\chi^2/$dof of $1.15$. 
There is thus an internal tension in the eBOSS dataset, when compared to our model, driven by the lowest two redshift bins.
We discuss possible reasons for this tension in Section~\ref{sec:tension} and compare to the results of earlier analyses further in Section~\ref{sec:comparison}.
It is important not to over-interpret the posterior constraints from the $z=2.2-4.6$ chain.
When there is an internal tension in the data and the overall reduced $\chi^2$ is poor, the posteriors can be driven by noise in the dataset and may not be meaningful.

We also checked for other redshift bins where the fit is poor.
Visually, Figure~\ref{fig:fps_data} suggests that the fit is poor for $z=4.0$ and $z=4.2$.
The reduced $\chi^2$ in Table~\ref{table:chi2} is moderately higher than $1$, but not significantly so, as the statistical errors are also large.
It is possible that some element of the covariance matrix, theoretical or systematic, is moderately underestimated in these bins.
The $z=2.6$ has a slightly elevated $\chi^2$/dof in the reduced redshift range chains, which may suggest that whatever causes the low redshift tension still has a small effect at $z=2.6$.  
%We ran chains which fit to each redshift bin individually, finding that the lowest two redshift bins were still discrepant with the others.

\begin{table}
	\centering
     \def\arraystretch{1.2}
     \begin{tabular}{|c|c|c|c|c|c|c|c|}
		\hline
		Redshift & $4.6$ & $4.4$ & $4.2$ & $4.0$ & $3.8$ & $3.6$ & $3.4$\\
		\hline
        FPS $z= 2.2-4.6$ & $1.08$ & $1.03$ & $1.46$ & $1.79$ & $0.87$ & $0.51$ & $0.63$ \\
        FPS $z= 2.6-4.6$ & $1.22$ & $0.85$ & $1.20$ & $1.40$ & $0.74$ & $0.36$ & $0.69$\\
        FPS $+ T_0$ $z= 2.6-4.6$ & $1.20$ & $0.86$ & $1.26$ & $1.43$ & $0.74$ & $0.32$ & $0.69$ \\
        \hline
        Redshift & $3.2$ & $3.0$ & $2.8$ & $2.6$ & $2.4$ & $2.2$ & Total \\
		\hline
        FPS $z= 2.2-4.6$ & $0.83$ & $1.82$ & $1.64$ & $0.98$ & $0.99$ & $1.36$ & $1.15$ \\
        FPS $z= 2.6-4.6$ & $0.76$ & $1.35$ & $1.29$ & $1.45$ & $2.97$ & $4.17$ & $1.03$ \\
        FPS$ + T_0$ $z= 2.6-4.6$ & $0.82$ & $1.30$ & $1.19$ & $1.59$ & $3.28$ & $4.48$ & $1.04$\\
  \hline
	\end{tabular}
    \caption{\label{table:chi2}
    $\chi^2$ per degree of freedom for the flux power spectrum for each redshift bin. \textbf{There are $35$ k-bins per redshift bin and $14$ parameters, and so $21$ degrees of freedom.}
    Shown is the likelihood at the best fit parameters in each chain.
    We show chains fitting to the flux power spectrum only at $z=2.2-4.6$, fitting to the flux power spectrum only at $z=2.6-4.6$, and fitting to the flux power spectrum and mean IGM temperature at $z=2.6-4.6$.
    The column labelled `Total' is the total $\chi^2$ per degree of freedom for the redshift bins in the fit (i.e., excluding $z=2.2$ and $z=2.4$ for the last two chains).}
\end{table}

Posterior constraints from the reduced redshift range flux power spectrum data show a mean optical depth in good agreement with other measurements.
As a reminder, these parameters measure deviations from the mean flux relation of Ref.~\cite{2007MNRAS.382.1657K}, so a value of $\tau_0=1$ and $d\tau_0=0$ corresponds to agreement with that model.
The best fit mean flux at $z=3$, $\tau_0 = 1.1$, and the redshift variation $d\tau_0$ is consistent with $0$.
This implies a mean optical depth at $z=3$ of $\tau^{\text{eff}}_{\text{H~{\sc i}}}(z=3) = 0.398$, which is extremely close to the best-fit value of $\tau^{\text{eff}}_{\text{H~{\sc i}}}(z=3) = 0.4$ from Ref.~\cite{2013MNRAS.430.2067B}, and consistent within the error bars with $\tau^{\text{eff}}_{\text{H~{\sc i}}}(z=3) = 0.36 \pm 0.1$ from Ref.~\cite{2007MNRAS.382.1657K}. 

\textbf{The spectral index, $n_P$, is $n_P > 0.969$ when using only the flux power spectrum. The $1\sigma$ boundaries hit the upper limit of the simulation prior. Our analysis does not rule out $n_P > 1$, although it is almost enclosed within the $1-\sigma$ contours. On inclusion of the mean temperature data, we find slightly tighter constraints of $n_P = 0.964 - 0.992$ at 68\% confidence.
These values are consistent with the Planck measurement of $n_s=0.965 \pm 0.004$ \cite{2020A&A...641A...6P}.}
The growth factor, $\Omega_M h^2$, is weakly constrained and not strongly affected by including mean temperature data.
Planck found $\Omega_M h^2 = 0.1424\pm0.001$, which is close to the posterior of our chains.
The power spectrum amplitude is $A_p/10^{-9} = 1.56 \pm 0.12$ for the flux power spectrum reduced redshift result.
The inclusion of the mean temperature shrinks the constraints moderately and shifts the posterior value down by about $1-\sigma$, \textbf{driven by a correlation with $\alpha_q$}.
Table~\ref{table:parameters} shows $A_s$, the power spectrum amplitude measured on large scales, which is related to $A_P$ via:
\begin{equation}
    A_s = \left(0.4/2\pi\right)^{n_P-1} A_P\,.
\end{equation}

We find $A_s = (1.67 \pm 0.12) \times10^{-9}$ for the flux power spectrum alone and $A_s = (1.54 \pm 0.012 )\times10^{-9}$ when including the mean IGM temperature.
Planck \cite{2020A&A...641A...6P} found a value of $A_s = \left(2.101^{+0.031}_{-0.034}\right)\times10^{-9}$. 
We also derived the value of $\sigma_8$ implied by our parameters by using CLASS in post-processing \cite{2011arXiv1104.2932L}. 
For the flux power spectrum alone, we find $\sigma_8 = 0.733 \pm 0.026$, and when the mean IGM temperature is included, $\sigma_8 = 0.704 \pm 0.025$ (see Table~\ref{table:parameters}).
The Planck result is $\sigma_8 = 0.811 \pm 0.006$ \cite{2020A&A...641A...6P}.
We thus measure a power spectrum amplitude around $3-\sigma$ lower than Planck or ACT CMB lensing \cite{2023arXiv230405202Q}.
Interestingly, the dark energy survey year 3 results measure $\sigma_8 = 0.733^{+0.039}_{-0.049}$ \cite{2022PhRvD.105b3520A}, in good agreement with our results. Other small-scale structure probes vary \cite[e.g.~][]{2020JCAP...05..042I, 2022JHEAp..34...49A, 2023JCAP...04..057Y}. 

Figure~\ref{fig:fps_data} shows the \lya forest flux power spectrum from \cite{2019JCAP...07..017C}, along with their estimated one sigma uncertainty (black).
Also shown are predictions from our multi-fidelity emulator based on the maximum posterior input parameters from MCMC analysis with only the \lya forest flux power emulator in the full and reduced redshift ranges, and MCMC analysis using both the mean temperature and flux power emulators.
The correlation between \lya and Si~{\sc iii} absorption is visible in the form of regular oscillations in the power spectrum (in Section~\ref{sec:likelihood} we describe the correction we make for Si~{\sc iii}).
The best-fit flux power spectrum is not significantly affected by the inclusion of the $T_0$ data in the likelihood. %, as is expected if the $T_0$ data mainly breaks degeneracies.

% --------------------------------------------------------------------------------------------------
% --------------------------------------------------------------------------------------------------

\subsection{Reionization and Other Parameters}\label{sec:astro}

\begin{figure}
    \centering
    \includegraphics[width=0.85\textwidth]{figures/astro_corner.pdf}
    \caption{\label{fig:astro_corner}
    Posterior constraints for the parameters of the helium reionization model ($z_i^{HeII}$, $z_f^{HeII}$, $\alpha_q$), the hydrogen reionization model ($z^{HI}$), the strong absorber models ($\alpha_{LLS}$, $\alpha_{DLA}$), the Silicon III correction ($f_\mathrm{SiIII}$), and the velocity to distance scale parameter $v_{scale}$.
    \textbf{Results are from the same three MCMC chains as Figure~\ref{fig:cosmo_corner}. `FPS $z=2.2-4.6$' (gold) uses the full redshift range eBOSS flux power spectrum dataset, `FPS $z=2.6-4.6$' (blue) uses a reduced redshift range eBOSS dataset flux power spectrum dataset, which removes the internal tension (Section~\ref{sec:tension}). The third chain, `FPS $+ T_0, z=2.6-4.6$' (red), uses the limited range eBOSS dataset but adds the mean IGM temperature constraints.}
    }
\end{figure}

\begin{figure}
    \centering
    \includegraphics[width=0.75\textwidth]{figures/t0_best_fit.pdf}
    \caption{\label{fig:temp_data}
    IGM mean temperatures from \protect\cite{2021MNRAS.506.4389G} (black lines and circles, with shading corresponding to one sigma uncertainty).
    Specifically their temperatures derived from the flux power spectrum calculated using high resolution \lya forest spectra.
    Also shown are predictions for the mean temperature from our multi-fidelity emulator corresponding to the same three maximum posterior input parameters used in Figure~\protect\ref{fig:fps_data}.
    }
\end{figure}

Figure~\ref{fig:astro_corner} shows the other parameters of our model from the same chains as Figure~\ref{fig:cosmo_corner}.
These are: three parameters of the helium reionization model (z$^{\text{He~{\sc ii}}}_i$, z$^{\text{He~{\sc ii}}}_f$, and $\alpha_q$), the midpoint of H~{\sc{i}} reionization z$^{\text{H~{\sc i}}}$, the parameters of the strong absorber model ($\alpha_{LLS}$, $\alpha_{DLA}$), the strength of the metal contamination $f_{SiIII}$ and the box velocity scale $v_{scale}$.

All data prefers an early start to helium reionization, $z_i^{HeII} > 3.95$ at $95\%$ confidence.
Interestingly, this is in agreement with constraints from the helium \lya forest, where regions of high transmission suggest that HeII reionization has already started at $z = 3.5$ \cite{2016ApJ...825..144W, 2021ApJ...912...38M}.
The mean temperature data reinforce this, but do not substantially shrink constraints, perhaps because the highest redshift mean IGM temperature we fit to is $3.8$, and the flux power spectrum already constrains reionization to start by this point.
The end of helium reionization, $z_f^{HeII}$, is \textbf{similarly constrained by the flux power spectrum data alone to be $z_f^{HeII} < 2.76$, and adding the mean IGM temperature data again reinforces this}.
This is consistent with the He~{\sc{ii}} \lya forest, which suggests an end at $z \leq 2.7$ \cite{2009ApJ...704L..89M, 2011ApJ...733L..24W, 2019ApJ...875..111W}.
\textbf{Our constraints on the timing of helium reionization hit our prior volume. However, in this case these are often driven more by the redshift limits of our datasets than our simulation volume. For example, the upper prior limit on $z^{HeII}_i$ is $4.1$, which is larger than the highest redshift IGM temperature data.}

The most significant effect of the mean IGM temperature data is on the spectral index during helium reionization, $\alpha_q$.
Smaller values of $\alpha_q$ correspond to a larger heating rate.
The flux power spectrum data prefers a high value of $\alpha_q$, although the constraints are weak.
However, as shown in Figure~\ref{fig:temp_data}, this high value of $\alpha_q$ produces a mean IGM temperature which is low and in disagreement with the data from Ref.~\cite{2021MNRAS.506.4389G}. \textbf{The posterior constraint on $\alpha_q$ when the mean temperature data is included is lower than that with the flux power spectrum data alone by about $4-\sigma$.}
Figure~\ref{fig:fps_data} shows that the flux power spectrum is not significantly different at the maximum posterior parameters of either chain.
Appendix~\ref{sec:t0-only} shows the results of chains which include only the mean temperature data.
They are consistent with the combined chains, and the helium reionization parameters are constrained at similar values.

\textbf{There are two possible explanations for this discrepancy in $\alpha_q$. The first is that some (hypothetical) marginalisation effect in the relatively weak constraints from the flux power spectrum increases the maximum likelihood $\alpha_q$. A second possibility is that the discrepancy is caused because the mean IGM temperature constraints assumed the Planck value of $\sigma_8$, which is inconsistent with our results. Assuming a larger value for $\sigma_8$ will increase the flux power spectrum and require a greater degree of thermal heating to smooth the gas. The mean IGM temperature constraints from Ref.~\cite{2021MNRAS.506.4389G} may thus be biased high. A consistent analysis of the flux power spectra from both eBOSS and XQ-100 varying $A_P$ and $\alpha_q$ simultaneously would resolve this question, and we will perform it in future work.}

The midpoint of hydrogen reionization is poorly constrained in all models.
The redshift range explored here ($z=2.2-4.6$) is well after the completion of hydrogen reionization, even in models where it ends late.
All our chains suggest a midpoint z$^{\text{H~{\sc i}}} \gtrsim 7.2$ (at 95\% confidence).
This is within the range allowed by other experiments: Planck suggests z$^{\text{H~{\sc i}}} = 7.7 \pm 0.7$, while an analysis of the \Lya emitter luminosity function finds z$^{\text{H~{\sc i}}} \sim 7.25$ \cite{2021ApJ...919..120M}.

We show results for the nuisance parameters associated with the strong absorbers and the \Lya-SiIII cross-correlation.
As discussed in Ref.~\cite{2023simsuite}, our simulation suite includes strong absorbers self-consistently using a galaxy formation model.
The strong absorber parameters measure the difference between the strong absorber model in the simulation and the model in the observed spectra, so that $\alpha_{LLS} = 0$ means that our galaxy formation model is a good match to the circumgalactic gas in the observed Universe.
DLAs are subtracted from both the observed and simulated spectra, and so $\alpha_{DLA}$ measures primarily the efficiency of the observational DLA finder. 
All our chains produce a posterior $\alpha_{DLA}$ tightly peaked and close to $0$.
The chains using the flux power alone are centered on $\alpha_{DLA} = 0$, while the chain including the mean temperature prefers a slightly negative value, perhaps indicating that the eBOSS DLA finder includes some false positives\footnote{\textbf{As DLAs finders are sensitive to reduced flux transmission, regions with more \Lya forest absorption are more likely than average to be flagged as a DLA. The efficiency of the DLA finder may thus affect the flux power spectrum}.}.
Note that DESI includes improved DLA finding algorithms based on machine learning \cite{Ho:2020,Parks:2018, 2022arXiv220100827W}.

For the restricted redshift chains, we measure $\alpha_{LLS} \sim 0.16$, while for the full redshift range $\alpha_{LLS} \sim 0.04$.
The preference for a non-zero $\alpha_{LLS}$ in the reduced redshift chains suggest that our simulations have fewer LLS than the real Universe.
LLS are on the boundary of being optically thick and thus radiative transfer effects within the gas are important, making them the most difficult absorbers to model accurately.
$\alpha_{LLS}$ can affect the flux power spectrum normalisation, and so it is reasonable to interpret the preference of the full redshift chains for a low $\alpha_{LLS}$ as an artifact of the fit.
However, it is also possible that the low $\alpha_{LLS}$ points to the origin of the internal tension, a possibility we discuss further in Section~\ref{sec:tension}. 

The SiIII cross-correlation is $f_{SiIII} = 0.0095 \pm 0.001$ from our $z=2.6 - 4.6$ chains.
The full redshift range prefers a slightly lower value of $0.0085 \pm 0.001$, which is in good agreement with the measurement of $0.008 \pm 0.001$ from DR9 by Ref.~\cite{2013A&A...559A..85P}.
The effect of $f_{SiIII}$ can be seen in the oscillations of the flux power spectrum in Figure~\ref{fig:fps_data}.
The results for $v_\mathrm{scale}$ are dominated by the prior, as expected \cite{2015JCAP...11..011P, 2020JCAP...04..038P}.
Constraints are weaker than for the simulated data, likely because the simulated data did not include noise and so was over-fitting.

Figure~\ref{fig:temp_data} shows the IGM mean temperature from the flux power spectrum on small scales from Ref.~\cite{2021MNRAS.506.4389G}.
Also shown in Figure~\ref{fig:temp_data} are predictions from our multi-fidelity emulator based on the maximum posterior input parameters from the same chains used in Figure~\ref{fig:fps_data}.
Once the mean temperature data is included in the fit, the chains are in good agreement.
However, when it is not included the chains prefer a lower mean IGM temperature, \textbf{as discussed above}.
The thermal history preferred by the full redshift range of the flux power spectrum is very similar to that preferred by the restricted range flux power spectrum. 

\subsection{Parameter Correlations}
\label{sec:correlations}

\begin{figure}
    \centering
    \includegraphics[width=0.98\textwidth]{figures/correlation_z26_46_t0.pdf}
    \caption{\label{fig:correlations}
    Correlation matrix between parameters for the chain using the flux power spectrum and mean IGM temperature for $z: 2.6-4.6$.
    }
\end{figure}
% Correlation matrix without T0
%           ns :  1.0000  0.2989 -0.0268 -0.0490  0.0257 -0.0754 -0.0491 -0.1703 -0.6291  0.3609  0.4835  0.1507 -0.0127
%           Ap :  0.2989  1.0000 -0.1286 -0.0670  0.2543  0.1259 -0.3552 -0.1103 -0.7505  0.1851 -0.4653  0.5361  0.0057
%        herei : -0.0268 -0.1286  1.0000  0.1633 -0.0118 -0.0476  0.0601  0.0165  0.0591 -0.2435  0.0829 -0.1175  0.0110
%        heref : -0.0490 -0.0670  0.1633  1.0000  0.0918 -0.0176 -0.0714 -0.0644  0.0628 -0.3173  0.0452 -0.1602 -0.0054
%       alphaq :  0.0257  0.2543 -0.0118  0.0918  1.0000  0.1144 -0.0463  0.0042 -0.0033 -0.1293 -0.1642  0.2450  0.0258
%          hub : -0.0754  0.1259 -0.0476 -0.0176  0.1144  1.0000 -0.1717  0.0392  0.1245 -0.1800 -0.0951 -0.1156  0.1681
%     omegamh2 : -0.0491 -0.3552  0.0601 -0.0714 -0.0463 -0.1717  1.0000  0.0952 -0.0590  0.1020  0.1259 -0.1542 -0.0753
%     hireionz : -0.1703 -0.1103  0.0165 -0.0644  0.0042  0.0392  0.0952  1.0000  0.1615  0.2233 -0.0266 -0.0099  0.0090
%         tau0 : -0.6291 -0.7505  0.0591  0.0628 -0.0033  0.1245 -0.0590  0.1615  1.0000 -0.4760 -0.0468 -0.2471  0.0572
%        dtau0 :  0.3609  0.1851 -0.2435 -0.3173 -0.1293 -0.1800  0.1020  0.2233 -0.4760  1.0000  0.3109 -0.0607 -0.0532
%        a_lls :  0.4835 -0.4653  0.0829  0.0452 -0.1642 -0.0951  0.1259 -0.0266 -0.0468  0.3109  1.0000 -0.6318 -0.0396
%        a_dla :  0.1507  0.5361 -0.1175 -0.1602  0.2450 -0.1156 -0.1542 -0.0099 -0.2471 -0.0607 -0.6318  1.0000  0.1124
%       fSiIII : -0.0127  0.0057  0.0110 -0.0054  0.0258  0.1681 -0.0753  0.0090  0.0572 -0.0532 -0.0396  0.1124  1.0000
% 
% Correlation matrix with T0
%           ns :  1.0000  0.2312 -0.0513 -0.0483  0.0943 -0.1385  0.0504 -0.1548 -0.6190  0.3904  0.6101  0.0683 -0.0274
%           Ap :  0.2312  1.0000 -0.1523  0.0397  0.3556  0.0219 -0.3646 -0.1661 -0.6959  0.1815 -0.3378  0.4202  0.0208
%        herei : -0.0513 -0.1523  1.0000 -0.2933 -0.3919 -0.0280  0.0721  0.4125  0.0368 -0.0293  0.0684 -0.1170  0.0152
%        heref : -0.0483  0.0397 -0.2933  1.0000  0.5297  0.1051 -0.0759  0.0796  0.1245 -0.1632 -0.0570  0.0413  0.0479
%       alphaq :  0.0943  0.3556 -0.3919  0.5297  1.0000  0.0539 -0.1226 -0.3059 -0.1224 -0.1207 -0.1296  0.2018  0.0096
%          hub : -0.1385  0.0219 -0.0280  0.1051  0.0539  1.0000 -0.3537  0.0459  0.2918 -0.2468 -0.0473 -0.1885  0.2780
%     omegamh2 :  0.0504 -0.3646  0.0721 -0.0759 -0.1226 -0.3537  1.0000  0.0606 -0.1742  0.1232  0.1544 -0.1543 -0.1355
%     hireionz : -0.1548 -0.1661  0.4125  0.0796 -0.3059  0.0459  0.0606  1.0000  0.1519  0.1456  0.0068 -0.0890  0.0370
%         tau0 : -0.6190 -0.6959  0.0368  0.1245 -0.1224  0.2918 -0.1742  0.1519  1.0000 -0.5392 -0.2271 -0.1165  0.0841
%        dtau0 :  0.3904  0.1815 -0.0293 -0.1632 -0.1207 -0.2468  0.1232  0.1456 -0.5392  1.0000  0.4176 -0.1630 -0.0950
%        a_lls :  0.6101 -0.3378  0.0684 -0.0570 -0.1296 -0.0473  0.1544  0.0068 -0.2271  0.4176  1.0000 -0.5910 -0.0462
%        a_dla :  0.0683  0.4202 -0.1170  0.0413  0.2018 -0.1885 -0.1543 -0.0890 -0.1165 -0.1630 -0.5910  1.0000  0.0880
%      fSiIII : -0.0274  0.0208  0.0152  0.0479  0.0096  0.2780 -0.1355  0.0370  0.0841 -0.0950 -0.0462  0.0880  1.0000

Figure~\ref{fig:correlations} shows the correlations between our parameters, for the chain using the flux power spectrum from $z=2.6 - 4.6$, as well as the mean IGM temperature.
Most correlations are weak.
We have deliberately chosen our pivot scale of $0.78$ Mpc$^{-1}$ to minimise the correlation between $A_P$ and $n_P$, and the correlation matrix confirms it is weak, with a correlation coefficient $r=0.2$. 
%Several of the stronger correlations arise from intrinsic degeneracies in the definition of the parameters and could in principle be reduced by modest redefinitions of pivot scales or pivot redshifts. 
There is a correlation between $\tau_0$ and $d\tau_0$ ($r=-0.54$), as the redshift bin which provides the strongest constraints on the optical depth is not exactly $z=3$.
The optical depth $\tau_0$ is anti-correlated with both $A_P$ ($r=-0.7$) and $n_P$ ($r=-0.62$) as its main effect is to change the amplitude of the flux power spectrum. 

There is a three-dimensional degeneracy between $\alpha_q$, $z_i^{HeII}$ and $z_i^{HeII}$ (see Figure~\ref{fig:correlations}), which allows a wide range of $\alpha_q$ to fit the flux power spectrum data, and is only broken by information from the thermal history.
Lower $\alpha_q$ corresponds to more heating from quasars during He~{\sc{ii}} reionization.
If He~{\sc{ii}} reionization starts earlier or ends later, the IGM requires more heating from quasars to match the observations, while the opposite is true for late starting, or early ending He~{\sc{ii}} reionization.
Appendix~\ref{sec:t0-only} shows the results of chains which include only the mean temperature data, which clearly shows this three-dimensional degeneracy: a slightly later start to helium reionization would require less total heating and thus a higher value of $\alpha_q$. 
Several of these correlations could be broken by the inclusion of higher redshift thermal history data, or lower redshift flux power spectrum data. 

Finally, the abundance of Lyman Limit Systems, $\alpha_{LLS}$, exhibits several interesting correlations.
$\alpha_{LLS}$ is anti-correlated with $\alpha_{DLA}$ ($r= - 0.59$), as the flux power spectrum templates for strong absorbers have similar shapes in neighbouring column density bins.
$\alpha_{LLS}$ is also correlated with $n_P$ ($r=0.61$) and $d\tau_0$ ($r=0.42$), due to similarities in the shapes of their flux power spectrum templates.
The combination of the three-way correlation between $n_P$, $\alpha_{LLS}$ and $\tau_0$ is exploited by the chains to explain the inconsistent $z=2.2, 2.4$ flux power spectrum bins and drives the discrepant constraints these chains show.
This correlation may be reduced by the inclusion of extra small-scale data available in the DESI early data release.

%--------------------------------------------------------------------------------------------------
\section{Discussion}
\label{sec:discussion}

In this Section we discuss the implications of the results in Section~\ref{sec:results}. 
Section~\ref{sec:tension} discusses possible explanations for the internal tension in the data between $z = 2.2 - 2.4$ and $z \geq 2.6$.
Section~\ref{sec:comparison} compares our results to other datasets and earlier \Lya~analyses.
Section~\ref{sec:altlikelihood} discusses how our results are affected by modifications to our likelihood.

\subsection{The Tension in the Lowest Redshift Bins}
\label{sec:tension}

\begin{figure}
    \centering
    \includegraphics[width=0.45\textwidth]{figures/lymandata-z2.2.pdf}
    \includegraphics[width=0.45\textwidth]{figures/lymandata-z2.6.pdf}
    \caption{\label{fig:p1d_data}
    Observational 1D power spectrum data from SDSS DR14 (solid black) \protect\cite{2019JCAP...07..017C}, SDSS DR9 (green dotted) \protect\cite{2013A&A...559A..85P} and KODIAQ/SQUAD \protect\cite{2022MNRAS.509.2842K}.
    Filled bands show the range covered by diagonal elements of the covariance matrix.
    (Left) At $z=2.2$.
    (Right) At $z=2.6$.}
\end{figure}

In this Section, we discuss possible explanations for the internal tension between the flux power spectrum data at $z=2.2 - 2.4$ and $z \geq 2.6$.
There are two generic possibilities: either an important physical effect is missing from our simulation model, or there is a systematic in the dataset not captured by the systematic error budget. 

To evaluate the possibility of systematic error, we can look at independent measurements of the flux power spectrum on similar scales and at similar redshifts.
Figure~\ref{fig:p1d_data} shows different measurements of the 1D flux power, $P_F(k)$, at $z=2.2$ and $z=2.6$.
We show the results from SDSS DR14 \cite{2019JCAP...07..017C}, SDSS DR9 \cite{2013A&A...559A..85P}, a recent analysis using high resolution spectra from KODIAQ/SQUAD \cite{2022MNRAS.509.2842K}, and from DESI Early Data Release data (DESI EDR) \cite{2023arXiv230606316G}.
At $z=2.6$ (and higher redshift bins) all analyses are in reasonably good agreement, given their respective statistical errors.
However, this is not the case at $z=2.2$, where there is some discrepancy between SDSS (DR14 and DR9), DESI and KODIAQ.
SDSS DR14 and DR9 are in good agreement, and in Appendix \ref{sec:dr9_results} we show that the posterior parameter constraints also agree well. 

The KODIAQ flux power spectrum is lower by around $1-\sigma$ on the smallest scales measured by eBOSS, which could be due to the effect of continuum modelling in the KODIAQ data \cite{2022MNRAS.509.2842K}.
The DESI EDR data agrees well with eBOSS for $k > 0.01$ s/km, but is discrepant by $> 2 \sigma$ for $k < 0.01$ s/km.
This discrepancy is also present at $z=2.4$, and is discussed in the DESI EDR papers, see Ref.~\cite{2023arXiv230606311R}, Appendix D.
They ascribed $30\%$ of the difference between eBOSS and DESI to continuum fitting, but the origin of the rest is currently unclear.
Given the fairly large disagreements between different measurements at $z=2.2$, systematic error is a highly plausible explanation for the internal tension we find.
In future work we will combine our likelihood function with the DESI flux power spectra and investigate their cosmological implications.

We should also consider possible theoretical explanations.
Explanations rooted in alternative early Universe models seem a priori unlikely as they would have to cause an effect only for $z < 2.6$, when the Universe is known to be matter dominated. 
However, there are a few possible astrophysical explanations.
Feedback effects from AGN become increasingly important at low redshift.
It is possible that a stronger AGN feedback prescription than we use, or than is implemented in current cosmological simulation suites, could efficiently disrupt gas at $z < 2.6$ on small scales and explain these results.
Interestingly, such an AGN feedback model has recently been proposed as an explanation for the low value of $S_8 = \sigma_8 (\Omega_M/0.3)^{0.5}$ preferred by some weak lensing surveys \cite{2022MNRAS.516.5355A}, which matches our results.
However, Ref.~\cite{2023AJ....166..228T} examined a wide range of AGN feedback models, including some much more aggressive than those in PRIYA (or ASTRID).
None of these models can affect the $z=2.2$ flux power spectrum at the level required to explain these results (Tillman, private communication).

DLAs are important at low redshift, and do affect the slope of the flux power spectrum.
Our simulations include a population of DLAs in good agreement with observations \cite{2023simsuite}, which are masked using the same procedure as the observational pipeline.
We include a free parameter to model the residual power from any DLAs not detected by eBOSS, and the posterior value for this free parameter is consistent with zero.
\textbf{We considered separate constraints on the DLA efficiency at low and high redshift: $\alpha_{DLA}(z < 2.6)$ and $\alpha_{DLA}(z \geq 2.6)$. The high redshift DLA efficiency $\alpha_{DLA}(z \geq 2.6)$ remained consistent with zero. However, the low redshift parameter was negative; $\alpha_{DLA}(z < 2.6) = -0.0134 \pm 0.0074$, indicating an excess of large-scale power in eBOSS at $z < 2.6$. While most cosmological parameters were unchanged, $n_P$ increased (by about $2-\sigma$) to $n_P = 0.916 \pm 0.014$, reducing the internal tension by $1/3$. Since this is only a partial resolution, the DLA parameter may be sensitive to a continuum fitting problem in the (relatively short) low redshift spectra.}

At low redshift, the \Lya~forest is increasingly contaminated by metal lines.
We include a simple prescription for \ion{Si}{III} and the inclusion of low redshift data does not drive the best-fit parameter for this model, preferring slightly less metal contamination.
However, it is possible that a more sophisticated model could help reduce the tension.

One interesting but entirely speculative possibility is suggested by the LLS abundance, $\alpha_{LLS}$, which is lower in the full redshift chains.
Ref.~\cite{Prochaska:2009a} identified a systematic in the SDSS colour selection which causes quasar sightlines containing Lyman Limit Systems (LLS) to be preferentially selected for spectroscopic followup.
Refs.~\cite{Worseck:2011, Fumagalli:2013} showed that, due to the width of the $u$-band filter in SDSS, LLS are over-sampled for $z=2.5-3.6$ for all quasars in the redshift range $z=3.0-3.6$.
It is thus possible that $\alpha_{LLS}$ depends on the quasar (not absorber) redshift.
Note that the flux power spectrum we measure depends on the absorber redshift and so a simple redshift split would not detect this effect\footnote{We ran a chain with two $\alpha_{LLS}$ parameters for $z < 2.6$ and $z \geq 2.6$. The maximum likelihood for $\alpha_{LLS}$ at $z > 2.6$ was $\sim 2-\sigma$ larger than in the full redshift chain, and $n_P$ increased by $\sim 0.5\sigma$, but all other parameters were unchanged.}.
A check for colour selection systematics would involve the flux power spectrum being computed from two different quasar redshift bins.
%The \Lya forest probes marginally higher over-densities at low redshifts, so it is possible that a novel self-interacting dark matter model could be constructed which explains these observations, but it would have trouble matching cluster data (commented because I think the model would be a bit contrived).

\subsection[Comparison of the Posterior Constraints to Other Lyman-alpha Analyses]{Comparison of the Posterior Constraints to Other \Lya Analyses}
\label{sec:comparison}

%Shift between DR9 and DR14 found in PD2020 but we don't (perhaps because the DLA model absorbs it).

%Other results: can compare only for the full redshift range.
%PD2020: ns low, sigma8 high, tau0 normal. We find tau0 high, sigma8 low, ns low. Quantitatively our results are more discrepant with Planck than theirs. Perhaps this is because their Taylor expansion tends to flatten gradients far from the central value, perhaps because we are measuring nP on small scales, not ns on large scales.
%PD2020 explicitly tested excluding the two lowest redshift bins and found no change in cosmological parameters. Not clear why.

It is interesting to compare the results of our chains to those of Ref.~\cite{2020JCAP...04..038P} (for DR14) and Ref.~\cite{2015JCAP...11..011P} (for DR9).
The most notable difference is that Ref.~\cite{2020JCAP...04..038P} tested excluding the lowest two redshift bins and found minimal change in the posteriors of their cosmological parameter.
This disagrees with our results.
We believe this discrepancy can be ascribed to our different treatment of nuisance parameters. Ref.~\cite{2020JCAP...04..038P} employed correction functions for supernova feedback from the OWLs simulation suite \cite{2013MNRAS.429.1734V} and for AGN feedback from the Horizon-AGN suite \cite{2020MNRAS.495.1825C}.
Each correction function is most significant at low redshift, and is included with a free amplitude parameter which is marginalised over.
In addition, earlier models were forced by computational limits to use the `splicing' technique of Ref.~\cite{2014JCAP...07..005B}.
In this model multiple simulation boxes with over-lapping scale ranges are combined to model the scales probed by the \Lya~forest.
A single larger simulation with $2048^3$ particles was used to generate a scale and redshift dependent correction function, and the amplitude of this correction was marginalised over with a Gaussian prior.

Our larger simulations can instead model all relevant scales in a single simulation and so do not need a free parameter for splicing.
In addition, we self-consistently incorporate models for stellar and AGN feedback and star formation into our simulations.
Thus the analysis of Ref.~\cite{2020JCAP...04..038P} differs from ours in that it has three nuisance parameters, each of which affects the lowest redshift bins most strongly and each of which is marginalised over in the chains.
Ref.~\cite{2015JCAP...02..045P} mentions that removing splicing reduces $n_s$ significantly (although the posterior value of the splicing correction is not reported), which is what we would expect if the splicing correction were absorbing an internal tension. 
Thus we believe that the $z=2.2$ and $z=2.4$ redshift bins contribute only marginally to the cosmological constraints in Ref.~\cite{2020JCAP...04..038P} and instead constrain splicing and AGN feedback. 
We should therefore compare the quantitative results of Ref.~\cite{2020JCAP...04..038P} to our reduced redshift chains. 

Ref.~\cite{2015JCAP...11..011P} find a mean optical depth of $\tau_{eff} (z=3) = 0.0025 \pm 0.0001$ and $d\tau = 3.734 \pm 0.015$ in DR9.
Ref.~\cite{2020JCAP...04..038P} do not report a value for $\tau_{eff}(z=3)$ from DR14, but we presume it is similar to that in DR9.
We define the mean optical depth relative to the power law $\tau_{eff} = 0.0023 (1+z)^{3.65}$, so that in our parameterization these constraints correspond to $\tau_0 = 1.09 \pm 0.05$ and $d\tau_0 = 0.084\pm 0.015$.
Their optical depth measurements are thus in good agreement with our measurements for the reduced redshift range of $z=2.6 - 4.6$. 

Ref.~\cite{2020JCAP...04..038P} found $n_s = 0.954 \pm 0.006$ for DR14 and $n_s = 0.938 \pm 0.010$ for DR9.
Meanwhile the power spectrum amplitude as measured by $\sigma_8$ is $0.826 \pm 0.02$ in DR14 and similar in DR9.
We find $n_P \sim 0.97$ and $\sigma_8 \sim 0.73$, \textbf{a $3.5-\sigma$ tension in $\sigma_8$.} Some of the differences between our constraints on $A_P$ and $\sigma_8$ are due to a lever arm effect: for $n_s < 1$ the power spectrum amplitude will be increased on larger scales and measuring $n_s$ will induce a correlation between $n_s$ and $\sigma_8$. \textbf{Translated into our parameters, the largest discrepancy lies in our measurements of the spectral slope. The ultimate source of this tension is not clear, but our simulation suite is substantially larger and more robust than that used by Ref.~\cite{2020JCAP...04..038P}. A possibility is that the splicing correction they use does not fully account for the correlation of small and large scales (for example, the accuracy of the splice may be cosmology dependent). Another possibility is that the emulator used is a polynomial expansion around a `best-fit' simulation, with $n_s = 0.9624$. This is quite far from the posterior constraints, and so the accuracy of the polynomial emulator may be reduced. A third possibility is the lack of an explicit model for LLS. There is a correlation between $n_P$ and $\alpha_{LLS}$, as shown in Appendix~\ref{sec:full_posteriors}, so that fixing $\alpha_{LLS} = 0$ could produce a low $n_P$.}

Ref.~\cite{2023arXiv230300746G} constrain the slope and amplitude of the linear matter power spectrum at $z=3$ using the reduced likelihood from Ref.~\cite{2023ApJ...944..223P}.
They use the full redshift range of the data, but without adding nuisance parameters.
Their best-fit parameters approximately correspond to $n_P \sim 0.93$ and $A_P \sim 1.5 \times 10^{-9}$, a low slope and low amplitude compared to Planck, and reasonably close to the results from our full redshift chain. \textbf{The differences between our respective $n_P$ constraints may be due to their model not including helium reionization effects.}

As shown in Appendix~\ref{sec:dr9_results}, we do not reproduce the $\sim 1\sigma$ shift in $n_s$ from DR9 to DR14 observed by Ref.~\cite{2020JCAP...04..038P}, and attributed by them to the different catalogues for masking DLAs and BAL.
Our DR9 and DR14 constraints are in very good agreement (as expected, since the two datasets have a large fraction of the spectra in common).
Our simulations include self-shielding of the gas following Ref.~\cite{Rahmati:2013} and a realistic DLA model, masked from the flux power spectrum in the same way as the observational data.
That the cosmological parameters are not affected by the change in DLA catalogue is reassuring, as it suggests that our analysis is indeed correctly marginalising over the uncertainty from DLAs.

%Notice that the average bias of forest gas without self-shielding is not the same as the average bias of forest gas with self-shielding and DLA masking \cite{2023MNRAS.524.1464P}. Furthermore, galaxy formation model changes which only affect dense gas cannot affect the \Lya flux power spectrum in our modelling and thus we are insensitive to the details of supernova feedback.

Rather than use effective broken power laws for the IGM thermal history as a function of redshift, we have explicit physical models for hydrogen and helium reionization, which include the scale-dependent effects of patchy reionization.
As shown in Figure~\ref{fig:temp_data}, the preferred IGM thermal history shows a temperature peak at $z=2.8$ and thus $T_0(z)$ cannot be described by a power law broken at $z=3$ as assumed in Ref.~\cite{2020JCAP...04..038P} and many earlier works.
Figure~\ref{fig:fps_data} shows that this does not affect the flux power spectrum on the scales measured by eBOSS.
However, DESI data probes smaller scales, so it is not clear that this will be the case in future. 

%We do not include special parameters for the running $\alpha_s$ and neutrino mass $M_\nu$ in our emulator, preferring to model the effective amplitude and slope of the primordial power spectrum on the scales probed by the \Lya~forest, similarly to the reduced likelihood of Ref.~\cite{2023ApJ...944..223P} (but with parameters for the thermal history).

%Comparison to PD 2020: different model, no Taylor expansion but a LHS. No splicing, but multi-fidelity emulator. Correction for Sn from OWLS and AGN from the different simulation Horizon-AGN. We integrate both directly into our simulations. Our simulations include DLAs, star formation, galactic winds, AGN explicitly. No quick lyman alpha. Explicit helium and hydrogen reionization models which add a scale-dependent bias. No special parameters for alpha and Mnu in the emulator, instead we model them as changing nP and AP respectively. 
%The same SDSS DR14 dataset has been analysed by Ref.~\cite{2020JCAP...04..038P}. There are many differences in approach between our modelling and the earlier simulation models, as well as improvements driven by improved computation. Most significantly, our multi-fidelity emulation scheme \cite{2022MNRAS.509.2551H} allows us to build a model for the \Lya~forest which fully resolves the forest in a $120$ Mpc/h box. 

%Similarly, we do not use the simple second-order gradient expansion of the dependence of the flux power spectrum on cosmological parameters used in Ref.~\cite{2020JCAP...04..038P} and earlier works, but instead build an emulator on a parameter grid sampled using a Latin Hypercube, which better captures correlations between parameters \cite{2019JCAP...02..050B}. 


\subsection{Likelihood Modifications}
\label{sec:altlikelihood}

We also considered modifications to the likelihood not shown here.
First, we increased the observational uncertainty from eBOSS by a uniform factor of two.
This increased the posterior uncertainties, but did not significantly resolve the internal tension at low redshift (as this is many $\sigma$).
We also considered removing the high redshift data, with $z > 3.8$, as is done in some earlier analyses \cite{2011MNRAS.413.1717B}.
We found that this made little difference as the statistical errors in the high redshift data are large and so they provide little information. 
%We did confirm that the high redshift bins are not in tension with the low redshift data. 
We considered removing the largest and smallest scales with cuts in $k$.
Several of the smallest scale bins are highly correlated, and so removing them either led to very poor constraints or had small effects, depending on the scale cut.
Removing the largest bins on the largest scales increased the posterior uncertainty, but did not noticeably shift the posteriors.
Thus none of these checks show any evidence for an internal tension between scales.

We tested whether a second mean flux rescaling slope would improve the fit to the observed \lya forest flux power (Figure~\ref{fig:fps_data}), especially at lower redshifts and smaller scales.
To do this, we added a second mean flux slope to the MCMC sampled parameters, and assigned each to a specific redshift range (we tested this using a redshift pivot of $z=3$ and $z=3.6$).
Posterior constraints on the other parameters from a chain run using the second mean flux slope were unaffected and the fit was not improved.

\section{Conclusions}\label{sec:conclusions}

We have developed a new likelihood and pipeline for the analysis of \Lya forest data and run MCMC chains using the Cobaya package \cite{2021JCAP...05..057T, 2019ascl.soft10019T}.
Our likelihood is built on a percent-level accurate emulator using the PRIYA simulations \cite{2023simsuite}.
We use the multi-fidelity emulation technique \cite{2022MNRAS.517.3200F} and a set of high resolution simulations to avoid the need to `splice' together multiple simulations resolving different scales.

We model the \Lya forest 1D flux power spectrum from eBOSS \cite{2019JCAP...07..017C}, and include several simulated and post-processed parameters.
Our main cosmological constraints are on the slope and amplitude of the \textbf{primeval} power spectrum on the scales probed by the \Lya forest ($n_P$ and $A_P$). 
We augment our \Lya flux power spectrum likelihood with information about the IGM thermal history \cite{2021MNRAS.506.4389G}.
With this information, we constrain the start, end, and heating rate for a patchy model of helium reionization (z$^{\text{He~{\sc ii}}}_i$, z$^{\text{He~{\sc ii}}}_f$, $\alpha_q$), as well as the mean optical depth and its evolution with redshift ($\tau_0$ and $d\tau_0$).
Our likelihood includes corrections to the \lya forest flux power spectrum from correlated Si~{\sc iii} absorption and from the presence of Damped \lya systems.

We found that the lowest redshift bins in the eBOSS flux power spectrum, at $z=2.2$ and $z=2.4$, produced results which were discrepant with those from higher redshifts.
The flux power spectrum from the DESI early data release is also discrepant with eBOSS at these redshifts.
It thus seems likely that this discrepancy is due to an as-yet unidentified systematic in the eBOSS pipeline at $z < 2.6$. We found that the discrepancy was reduced if the DLA finder efficiency was allowed to be redshift dependent. However, an unmodelled astrophysical effect, perhaps connected with AGN feedback, is still possible.  

When removing the lowest redshift bins from the analysis, we find, \textbf{for a primeval power spectrum with a pivot scale of $0.78$ h/Mpc}:
\begin{itemize}
    \item A power spectrum slope of $1.0 > n_P > 0.975$, in good agreement with Planck.
    The upper limit is set by the prior volume of our emulator.
    \item A power spectrum amplitude $A_p=\left(1.56\times10^{-9}\right) \pm 0.12$, which translates to $A_s=\left(1.67\pm0.13\times10^{-9}\right)$ or $\sigma_8 = 0.733 \pm 0.026$, approximately $2-3\sigma$ lower than measurements from Planck, but in agreement with some other measurements from weak lensing or galaxy surveys.
    \item \textbf{An early start and late finish to helium reionization, beginning at $z > 4.04$ and ending at $z < 2.7$. Our data is consistent with helium reionization still being underway at the lowest redshift  eBOSS data we use, $z=2.6$, and with it being already underway by $z=4.1$, above which the eBOSS constraining power is weak. Our chains suggest a low peak IGM temperature of $\sim 10^4$ K at $z\sim 2.8$.}
    \item Weak constraints on hydrogen reionization and the matter density $\Omega_M h^2$.
\end{itemize}

\textbf{We added mean IGM temperature data to our chains, improving constraints on the thermal history. Our best-fit parameters from the flux power spectrum alone prefer a peak IGM temperature data much lower than the constraints of Ref.~\cite{2021MNRAS.506.4389G}. It is possible that these constraints are affected by assuming the Planck value of $\sigma_8$, which is inconsistent with our results. Our most robust constraints are thus those from eBOSS alone.}

%Future work
In future work, we will combine our \Lya likelihood with other cosmological information, in particular the Planck CMB and Baryon Acoustic Oscillation measurements, with which we can constrain several extensions to the $\Lambda$CDM model.
While we defer quantitative constraints to later papers, we are able to qualitatively discuss the constraints we expect.
We will be able to constrain the running of the spectral index, $\alpha_s = \frac{d n_s}{d \ln k}$.
Constraints on $\alpha_s$ come from the difference between the spectral index measured by the CMB on large scales and the spectral index measured by the \Lya forest on small scales.
% As our $n_P$ constraints are completely consistent with Planck, the combined constraints are likely to be consistent with zero running.
% It is also likely that the strong constraints on early dark energy found by Ref.~\cite{2023arXiv230300746G} from the low $n_s$ in earlier analyses will be weakened.
The sum of neutrino masses can be constrained via a comparison between the power spectrum amplitude on CMB and \Lya scales \cite{2020JCAP...04..025P}. \textbf{Our constraints are in strong tension with those of Ref.~\cite{2020JCAP...04..038P}, which will certainly affect neutrino mass constraints from the \Lya forest.}
Since our preferred power spectrum amplitude is lower than that of Planck, we will likely have a preference for a non-zero neutrino mass, although the strength of the preference and the value preferred is yet to be determined. 

We will also incorporate new data sets into our likelihood.
We will examine the posterior parameter constraints from the DESI EDR flux power spectrum.
The statistical power of DESI EDR is currently weaker than that of eBOSS, but it is able to measure smaller scales ($k_F < 0.05$ s/km rather than $k_F < 0.02$ s/km for eBOSS).
The higher resolution data may also improve the internal consistency of the dataset at $z < 2.6$.
Finally, we can perform a joint analysis of eBOSS and the high resolution \Lya~forest flux power spectra from Ref.~\cite{2022MNRAS.509.2842K}.
These smaller scales would directly measure the parameters of helium reionization, without the intermediate step of the mean IGM temperature, allowing an end-to-end validation of the consistency of our modelling of large and small scales.


\acknowledgments
MAF is supported by a National Science Foundation Graduate Research Fellowship under grant No. DGE-1326120.
SB was supported by NSF grant AST-1817256 and by NASA-80NSSC21K1840. 
MQ was supported by NSF grant AST-2107821. 
MFH is supported by a NASA FINESST grant No. ASTRO20-0022.
Computing resources were provided by Frontera LRAC AST21005.
The authors acknowledge the Frontera computing project at the Texas Advanced Computing Center (TACC) for providing HPC and storage resources that have contributed to the research results reported within this paper.
Frontera is made possible by National Science Foundation award OAC-1818253.
URL: \url{http://www.tacc.utexas.edu}. Analysis computations were performed using the resources of the UCR HPCC, which were funded by grants from NSF (MRI-2215705, MRI-1429826) and NIH (1S10OD016290-01A1).

\appendix
%\section{Results for Alternative Likelihoods}\label{sec:alt_results}

% \subsection{Parameter Priors}\label{sec:priors}

% In this section we compare posteriors that use priors for black hole feedback, $\epsilon_{AGN}$, and the matter density, $\Omega_M h^2$, to our main results, which did not include parameter priors.
% From Planck \cite{2020A&A...641A...6P}, we have a prior for $\Omega_M h^2=0.1424\pm0.001$.
% The prior for the black hole feedback factor, $\epsilon_{AGN}$, is included to effectively remove this parameter from the inference, as it is not well constrained and does not have a strong effect on the flux power or mean temperature (the prior is $\epsilon_{AGN} = 0.05 \pm 0.01$).
% All priors are Gaussian and implemented in the likelihood.

% The chains are once again divided into redshift ranges, with one pair using $z=2.2-4.6$ and the other using $z=2.6-4.6$.
% Figure~\ref{fig:priors_corner} shows the results when using priors, as well as our main results without priors, for comparison.
% When we include the Planck prior on $\Omega_M h^2$ and the prior on $\epsilon_{AGN}$, all the other parameters are changed only marginally.
% This is encouraging, as our results are robust to the inclusion of priors, and the correlations between most sets of parameters are weak.
% The only parameter that does change is $h$, which for the full redshift range chain begins to develop a second mode in the posterior near $h=0.69$.
% This is likely due to the degeneracy between $h$ and $\Omega_M h^2$.

% \begin{figure}
%     \centering
%     \includegraphics[width=\textwidth]{figures/priors.pdf}
%     \caption{\label{fig:priors_corner}
%     Posteriors for chains run using priors on $\epsilon_{AGN}$ and $\Omega_M h^2$ (red, blue), as compared with their counterparts run without priors (black, yellow).
%     }
% \end{figure}

% --------------------------------------------------------------------------------------------------
% --------------------------------------------------------------------------------------------------
% --------------------------------------------------------------------------------------------------
% --------------------------------------------------------------------------------------------------
% --------------------------------------------------------------------------------------------------

% --------------------------------------------------------------------------------------------------
% --------------------------------------------------------------------------------------------------
% --------------------------------------------------------------------------------------------------
% --------------------------------------------------------------------------------------------------
% --------------------------------------------------------------------------------------------------


\section{Leave-one-out versus Emulator Error}
\label{sec:loovsgperr}
\begin{figure}
    \centering
    \includegraphics[width=\textwidth]{figures/loo_vs_emu_error_wlegend.pdf}
    \caption{\label{fig:loo_v_emu}
    Emulator error and leave-one-out errors across parameter space.
    For eight of the input parameters, the training samples (grey crosses for LF, red circles for HF), GP emulator errors (yellow dots), and scale-averaged leave-one-out errors (red dashed) are shown. \textbf{All chains contain the mean temperature data, and the eBOSS flux power spectrum from $z=2.6 - 4.6$.}
    Shown are 1D marginalised posteriors for the chains with the default likelihood (red) compared to chains run adding the GP emulator error to the likelihood (yellow).
    }
\end{figure}

In this Appendix we evaluate the impact of including the Gaussian Process interpolation error, $\boldsymbol{\sigma}_{GP}$, on the posterior parameters, as discussed in Equation~\ref{eq:covariance}.
Figure~\ref{fig:loo_v_emu} compares the effect of including emulator errors, showing the training samples, GP emulator errors and scale-averaged leave-one-out errors.
The leave-one-out error is independent of position in parameter space, whereas the GP error is larger towards the edge of parameter space.
\textbf{The largest effect is on the matter density $\Omega_M h^2$. With the GP error, helium reionization starts later and finishes earlier, as well as producing more heating. On the other hand, hydrogen reionization happens earlier. However, these shifts are all less than $1.5-\sigma$. The cosmological parameters are unchanged.}
%\begin{figure}
%    \centering
%    \includegraphics[width=\textwidth]{figures/2xboss.pdf}
%    \caption{\label{fig:2xboss_corner}
%    Posteriors for a chain run using observations with inflated errors: the default errors (black), root two times the default (red), and two times the default (yellow).
%    These use only the flux power, and include the full redshift range, $z=2.2-4.6$.
%    }
%\end{figure}


% --------------------------------------------------------------------------------------------------
% --------------------------------------------------------------------------------------------------
% --------------------------------------------------------------------------------------------------
% --------------------------------------------------------------------------------------------------
% --------------------------------------------------------------------------------------------------


% \subsection{Single-Fidelity Results}\label{sec:sf_results}

% \begin{figure}
%     \centering
%     \includegraphics[width=\textwidth]{figures/sfemu_corner.pdf}
%     \caption{\label{fig:sfemu_corner}
%     Posteriors for chains run with a single-fidelity (black) and a multi-fidelity (red) emulator.
%     Both chains use the flux power and mean temperature, and have no priors.
%     }
% \end{figure}

% In this section, we present results from a chain using a single-fidelity emulator.
% This predicts the flux power and mean temperature at the resolution of our LF simulation suite.
% Figure~\ref{fig:sfemu_corner} shows parameters posteriors for the single-fidelity emulator (black) and, for comparison, the multi-fidelity emulator (red).
% Parameters excluded from Figure~\ref{fig:sfemu_corner} were unaffected by the choice of single- or multi-fidelity.

% Thus, the parameters shown are all affected by this change, and include: the mean flux rescaling parameters, $n_P$, $A_p$, the start of He~{\sc ii} reionization, $\alpha_q$, $h$, $\Omega_M h^2$, and the midpoint of H~{\sc i} reionization.
% The first four of these are consistent with each other; the single-fidelity prefers a larger slope and smaller amplitude for the mean flux and primordial power.
% The two He~{\sc ii} reionization parameters and midpoint of H~{\sc i} reionization are correlated, so the shift in these may be due to those degeneracies.
% Also correlated are $h$ and $\Omega_M h^2$, which are both significantly affected in the single-fidelity chain.
% The main way in which $h$ alters the flux power is through the mean flux, shifting the matter power and thus the flux power.
% The change in $h$ is therefore not entirely surprising, given the change in the mean flux rescaling parameters.

% The results for the single-fidelity are similar to those when the lowest redshift is omitted, Figure~\ref{fig:zrange}, especially for $h$, $A_p$, and z$^{\text{H~{\sc i}}}$.
% The results for the single-fidelity are also similar to those when the largest scales are omitted, Figure~\ref{fig:krange}, especially for $h$, $n_P$, and $\Omega_M h^2$.
% In combination, this may indicate that the single-fidelity is underfitting the largest scales and lowest redshifts.

% --------------------------------------------------------------------------------------------------
\section{BOSS DR9 Data}\label{sec:dr9_results}
\begin{figure}
    \centering
    \includegraphics[width=\textwidth]{figures/dr9_allp_corner.pdf}
    \caption{\label{fig:dr9_corner}
    Posteriors for chains run using observations from the earlier SDSS data release, DR9, for the reduced redshift range (gold) and full redshift range (blue), compared to our main chains using DR14 with the reduced redshift range (black) and full redshift range (red).
    }
\end{figure}

In this section we compare posteriors obtained using a previous observational data set, specifically the flux power spectrum from \cite{2013A&A...559A..85P}, which is based on BOSS DR9 quasar spectra.
Figure~\ref{fig:dr9_corner} shows the posteriors for chains run with DR14, with both reduced and full redshift ranges, and chains with DR9, again with the reduced and full redshift ranges.

We show chains using only the flux power spectrum likelihood, to emphasise any differences between the datasets.
Differences between DR9 and DR14 are small, although, as expected, the posterior parameter ranges for DR9 are moderately larger than for DR14.
The largest parameter shift is $d\tau_0$, which shifts by $< 0.5 \sigma$. The shift from the omission of the lowest redshift bins is consistent between the two observational data sets.
Both exhibit the same internal tension.

\textbf{Although DR14 includes many more quasars, the covariance matrix is often dominated by systematic error. At $z=2.6$ DR14 and DR9 have very similar errors. DR14 has smaller measurement uncertainty at $z \geq 3$, but the lower redshifts have smaller errors and thus contain more information.}

\section{Mean Temperature Only Posteriors}\label{sec:t0-only}

\begin{figure}
    \centering
    \includegraphics[width=\textwidth]{figures/datasets_t0_corner.pdf}
    \caption{\label{fig:t0_datasets}
    Posteriors for chains run with only the mean temperature likelihood.
    Shown are chains using each of the four observational measurements of the mean temperature: using the flux power spectrum (blue), using the Doppler width distribution (BPDF, yellow), using the curvature statistic (red), and using a wavelet decomposition (black).
    The main results of this work use the flux power derived mean temperatures.
    }
\end{figure}

In this section, we present results from chains run using only the mean temperature likelihood.
Shown in Figure~\ref{fig:t0_datasets} are four chains, each using one of the observational mean temperatures, derived from different \lya forest summary statistics: the flux power spectrum, the Doppler width distribution, the curvature statistic, and a wavelet decomposition.
Most of the cosmology parameters are entirely unconstrained by the mean temperature and are omitted from the Figure.
We show $A_p$ for reference.
The three He~{\sc ii} reionization parameters are well constrained by the mean temperature history.
There is very little difference between the different mean temperature observations, with the BPDF derived temperature differing the most, specifically preferring a later start to He~{\sc ii} reionization, and less heating, corresponding to a lower mean IGM temperature.
However, differences are well within $1\sigma$.

The midpoint of H~{\sc i} reionization has a marginal preference for a late midpoint, but is very weakly constrained.
The mean temperature provides information on $z_{HI}$ as the IGM cools from the completion of H~{\sc i} reionization, setting the temperature before the onset of helium reionization.
Incorporating measurements of the mean IGM temperature at $z > 3.8$  \cite[e.g.~][]{2023arXiv230402038G} could substantially improve these constraints and we may do so in future work.

\section{Full Posteriors}
\label{sec:full_posteriors}
\begin{figure}
    \centering
    \includegraphics[width=\textwidth]{figures/allp_corner.pdf}
    \caption{\label{fig:full_posterior}
    Posteriors for the full set of simulation parameters, both cosmological and astrophysical.
    Shown are the same chains from the figures in Section~\ref{sec:results}, thus the correlations between the the parameters in Figure~\ref{fig:cosmo_corner} and those in Figure~\ref{fig:astro_corner} are the only new information here. \textbf{`FPS $z=2.2-4.6$' (gold) uses the full redshift range eBOSS flux power spectrum dataset, `FPS $z=2.6-4.6$' (blue) uses a reduced redshift range eBOSS dataset flux power spectrum dataset, which removes the internal tension (Section~\ref{sec:tension}).
    The third chain, `FPS $+T_0, z=2.6-4.6$' (red), uses the limited range eBOSS dataset but adds the mean IGM temperature constraints.}
    }
\end{figure}

Figure~\ref{fig:full_posterior} presents the full posteriors, including the correlations between the cosmology and astrophysics parameter sets, using the same chains discussed extensively in Section~\ref{sec:results}.
Correlations are discussed in Section~\ref{sec:correlations}.
%For the chains run using both the mean temperature and flux power, the main correlations are: a slightly negative correlation between $n_P$ and $A_p$; a negative correlation between z$^{\text{He~{\sc ii}}}_i$ and both $\alpha_q$ and z$^{\text{He~{\sc ii}}}_f$; a positive correlation between z$^{\text{He~{\sc ii}}}_i$ and z$^{\text{H~{\sc i}}}$; a negative correlation between $\alpha_q$ and z$^{\text{H~{\sc i}}}$; and a positive correlation between $\alpha_q$ and z$^{\text{He~{\sc ii}}}_f$.

%The only correlation between astrophysics and cosmology is a positive correlation between $A_p$ and $\alpha_q$.
%This may indicate that a larger $\alpha_q$ (which corresponds to less heating) is appropriate when the primordial power spectrum amplitude is larger, which leads to more structure.



% --------------------------------------------------------------------------------------------------
% --------------------------------------------------------------------------------------------------
% --------------------------------------------------------------------------------------------------
% --------------------------------------------------------------------------------------------------
% --------------------------------------------------------------------------------------------------


% \subsection{Reduced Redshift Range}\label{sec:reducedz}

% In this section, we present results from chains run using a reduced range of redshifts.
% Both chains are run using the multi-fidelity emulator, with both the mean temperature and flux power likelihoods.
% One chain is run with the lowest redshift bin ($z=2.2$) omitted, while another chain is run leaving out the three highest redshift bins ($z=4.2-4.6$).
% The results of these chains are shown in Figure~\ref{fig:krange} (red and yellow), along with a chain using the full range of redshifts (black).

% Most of the parameters are relatively stable between these two chains: $n_P$, $A_p$, the three He~{\sc ii} reionization parameters, $\Omega_M h^2$, and $\epsilon_{AGN}$.
% The midpoint of reionization redshift, z$^{\text{H~{\sc i}}}$, is affected by the choice of redshifts to include.
% When the high redshifts are omitted, the midpoint is pushed later, as information on the asymptotic cooling after reionization is lost.
% When the low redshift is omitted, the midpoint is also shifted lower, though this is likely through the degeneracy with the start of He~{\sc ii} reionization.

% When the low redshift is omitted, the value for $h$ shifts towards the middle of the range.
% This is an indication that the value for $h$ in Figure~\ref{fig:sfemu_corner} (single-fidelity, LF results) is partially driven by possible underfitting of the $z=2.2$ flux power and mean temperature.
% These results indicate that the full redshift range used in this work is worth including.

% \begin{figure}
%     \centering
%     \includegraphics[width=\textwidth]{figures/zscale_corner.pdf}
%     \caption{\label{fig:zrange}
%     Posteriors for chains run with reduced ranges for the redshifts used in the flux power and mean temperature likelihoods.
%     Shown are a chain without the lowest redshift bin (red), a chain without highest three redshift bins (yellow), and a chain using all redshifts (black), for comparison.
%     }
% \end{figure}

% --------------------------------------------------------------------------------------------------
% --------------------------------------------------------------------------------------------------
% --------------------------------------------------------------------------------------------------
% --------------------------------------------------------------------------------------------------
% --------------------------------------------------------------------------------------------------


% \subsection{Reduced Scale Range}\label{sec:reducedk}

% In this section, we present results from chains run using a reduced range of scales.
% Both chains are run using the multi-fidelity emulator, the full redshift range, with both the mean temperature and flux power likelihoods.
% One chain is run with the three largest scales (three smallest $k$) omitted.
% Note that this is roughly the scales where systematics in the continuum dominate over statistical error.
% Another chain is run leaving out the five smallest scales (five largest $k$).
% \spb{It would be nice to see a chain which excludes all data where the spectrograph resolution dominates. Unfortunately for $z = 2.4$, this is $k > 0.008672$ s/km, which is a lot of the data. Probably the constraints just become poor, but it would be nice to see it anyway.}.
% The results of these chains are shown in Figure~\ref{fig:krange} (red and yellow), along with a chain using the full range of scales (black).

% Several parameters are relatively stable between these two chains: $A_p$, the three He~{\sc ii} reionization parameters, and $\epsilon_{AGN}$.
% The value for $n_P$ is slightly higher when the largest scales are omitted, highlighting the importance of the large scales, and thus the volume used in our simulations.
% The midpoint of reionization redshift, z$^{\text{H~{\sc i}}}$, is not well constrained when the large scales are omitted, likely as a result of the higher value of $n_P$.
% When the large scales are omitted $\Omega_M h^2$ shifts off of the edge slightly, and this is accompanied by a secondary mode strengthening in $h$.
% This may indicate that the value for $h$ and $\Omega_M h^2$ are partially driven by overfitting of small scales, at the expense of large scales.
% These results indicate that the increased volume of the simulations used in this work have helped the analysis, by better resolving the largest scales.

% \begin{figure}
%     \centering
%     \includegraphics[width=\textwidth]{figures/kscale_corner.pdf}
%     \caption{\label{fig:krange}
%     Posteriors for chains run with reduced ranges for the scales used in the flux power likelihood.
%     Shown are a chain without the three largest scales (red), a chain without the five smallest scales (yellow), and a chain using all scales (black), for comparison.
%     }
% \end{figure}


% --------------------------------------------------------------------------------------------------

\bibliographystyle{JHEP.bst}
\bibliography{refs}

\end{document}
